\documentclass{school-22.211-notes}
\date{May 18, 2013}

\begin{document}
\maketitle

\lecture{Review: Ch 9 from Handbook of Nuclear Science \& Engineering}

\uline{\textbf{8 questions you should know from Ch 9}}:

\begin{enumerate}
\item LWR: roughly how many pins? how many axial zones a pin?

\bigskip 
  
  \item What are the typical steps in a modern lattice calculation to
  compute homogenized, two-group cross sections? 

\bigskip 

  
\item What are the main methods for resonance calculations? What are
  their pros and cons?

\bigskip 
  
  
\item How does NJOY work? What does xs in xs table depend on?
  Hint: infinitely dilute xs, background xs.

\bigskip 

  
\item How is resonance interference effect accounted for in lattice physics codes?

\bigskip 

  
\item What are some of the pros and cons of: CP, PN, SN, and MOC?

\bigskip 

  
\item How is leakage effect treated when we apply data generated by
  lattice physics code into nodal code?

 \bigskip 
 
  
\item What are the three most important burnable absorber materials in
  LWRs? Can you estimate how many nuclides we need to track in a 3D
  calculation?
\end{enumerate}


\iffalse
\clearpage
\textbf{Answers:}

\begin{enumerate}
\item 15,000 to 20,000 pins. Typical bundle has seven or eight axial zones.
\item Fig.3, also Fig.10 from p.58.
\item Ultrafine energy groups, equivalence theory, and subgroup
  method. p.61. Also see table 7 on p.64. p.111 for more limitations of equivalence theory. 
\item Eq. 3 on p.12, also section 1.4.1. 
\item 1.4.2, correction factor to apply on top of NJOY-generated xs.

\item p.20.

\item Fundamental mode p.23

\item Boron, gadolinium, erbium (p.251). $10^9$ (p.252).
  
\end{enumerate}
\fi


\clearpage
\end{document}

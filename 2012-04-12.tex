\documentclass{school-22.211-notes}
\date{April 12, 2012}

\begin{document}
\maketitle

\lecture{Exam 2 Review}
\topic{General Background}
\begin{enumerate}
\item Material $\kinf$ from group cross sections: 
  \begin{itemize}
  \item One-group: no flux dependency; cross sections that treat pin-cells as homogenized preserve exactly our MC fuel reactivity for infinite repeating lattices. 
  \eqn{ \kinf = \frac{\nu \bar{\Sigma}_f }{\bar{\Sigma}_a} }
  \item Two-group: Solve from two-group balance equation, still only depend on integrated cross section (we approximate the flux ratio with cross sections). 
    \eqn{ \kinf = \frac{\nu \bar{\Sigma}_{f1} + \nu \bar{\Sigma}_{f2} \frac{\hat{\Sigma}_{s12} }{\bar{\Sigma}_{a2}}}{\bar{\Sigma}_{a1} + \hat{\Sigma}_{s12} }   }
  \item Balance equation, \textcolor{red}{what is it?} 
  \end{itemize}
\item $\keff$: take into account leakage,
  \eqn{ \keff = \frac{\nu \Sigma_f}{DB^2 + \Sigma_a} } 
\item Materials/geometrical buckling
\item Infinite medium critical buckling
\item Eigenvalues/eigenfunctions
\item Fission yields
\item Transient fission product equations
\item I/Xe reactivity effects
\item Pm/Sm reactivity effects
\item Divergence theorem
\item Laplacian 
\end{enumerate}



\clearpage
\topic{Diffusion Theory} 
\begin{enumerate}
\item Transport cross section and diffusion coefficients: from the net current equation, assume the scattering is isotropic in the COM system, and use transport correction $P_0$ approximation, 
  \eqn{ D = \frac{1}{3 \Sigma_{tr} } }
  \eqn{ \Sigma_{tr} = \Sigma_t - \Sigma_{s1} = \Sigma_t - \frac{2}{3A} \Sigma_s }

\item Partial currents and albedos
\begin{enumerate}
\item Zero flux boundary condition:
  \eqn{ \phi (0) = 0 }

\item Zero incoming flux boundary condition (more accurate than zero flux, because there could still be flux leaking out the geometry):
\eqn{ \left. \psi(\vecr, E, \vecOmega) \right|_{\vecn \cdot \vecOmega < 0} = 0}
Integrating over all angles in half space, that is incoming partial current is zero, 
\eqn{ J^- (\vecr_i, E) = 0}
There are two formulism to solve this:
  \begin{itemize}
  \item Kord's formulism: 
    \eqn{ J^- (\vecr, E) &= \frac{1}{4} \phi(\vecr, E) - \frac{1}{2} J_n (\vecr, E)  = 0, & \Aboxed{ \frac{J}{\phi} &= \frac{1}{2} } }
  \item The more conventional approach is to approximate with extropolation boundary condition: 
    \begin{align}
      J^- &= \frac{1}{4} \phi + \frac{D}{2} \gradient \phi_n \\
      \Aboxed{ \frac{\gradient \phi}{\phi} &= - \frac{1}{2D} = - \frac{3\Sigma_{tr}}{2} = - \frac{1}{d_{\mathrm{extrap}}}} 
\mbox{   where } d_{\mathrm{extrap}} = \frac{2}{3 \Sigma_{tr}}  = \frac{2}{3} \lambda_{\mathrm{tr}}
    \end{align}
    where $D$ is the property of the material inside, and $\lambda_{tr}$ is transport mean free path. Notice that the exact coefficient in $d_{\mathrm{extrap}}$ may be different. 
  \end{itemize}

\item \hi{Albedo boundary condition} is the measurement of how much flux is reflected back: 
\eqn{ \alpha = \frac{J^- (\vecr_i, E)}{J^+ (\vecr_i, E)} }
\end{enumerate}

\item Effective down-scattering cross section: 
  \eqn{ \hat{\Sigma}_{s12} = \bar{\Sigma}_{s12} - \bar{\Sigma}_{s21} \frac{\phi_2}{\phi_1} }
  So that up-scattering is zero: 
  \eqn{ \Sigma_{21}  = 0 }

\item Removal cross section
  \eqn{ \Sigma_{rg} = \Sigma_{tg} - \Sigma_{sgg} = \Sigma_{ag} + \Sum_{g'=1, g'\neq g}^G \Sigma_{sgg'}  }

\item Two-group diffusion equations,
  \begin{align}
    \left[ \begin{array}{cc} 
        \frac{\nu \bar{\Sigma}_{f1}}{\kinf} -  \bar{\Sigma}_{a1}  - \hat{\Sigma}_{s12} & \frac{\nu \bar{\Sigma}_{f2}}{\kinf}   \\
        \hat{\Sigma}_{s12} &  - \bar{\Sigma}_{a2}  
      \end{array} \right] 
    \left[ \begin{array}{c} \phi_1 \\ \phi_2 \end{array} \right] = 0
  \end{align}

\item Multi-group diffusion equations
\item Interface and boundary conditions
\item Finite-difference diffusion equation
\item Fission source iterations
\item Gauss-Jacobi flux inner iterations
\item Dominance ratio estimation
\end{enumerate}

\clearpage
\topic{Practise Problems}
\begin{enumerate}
\item Homogeneous (single-region) diffusion problems.
\item Multi-region diffusion problems

\end{enumerate}


\end{document}

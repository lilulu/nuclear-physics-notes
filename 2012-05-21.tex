\documentclass{school-22.211-notes}
\date{May 23, 2012}

\begin{document}
\maketitle

%%%%%%%%%%%%%%%%%%%%%%%%% Final Exam Start %%%%%%%%%%%%%%%%%%%%%%%%%%%%
\lecture{Final Review}
\topic{Exam 1 Re-cap}
\begin{enumerate}
\item Basic Nuclear Data: 
\begin{itemize}
\item Know how to solve for density, cross section etc. 
\end{itemize}

\item Four pieces of nuclear physics: 
  \begin{itemize}
    \item Fission Spectra
    \item Elastic Scattering
    \item SLBW Resonance
    \item Scattering Kernals
  \end{itemize}

\item Homogeneous resonance self-sheilding
  \begin{itemize}
    \item Infinite media
    \item Dilution factors
    \item RI
  \end{itmemize}

\item Doppler effects
  \begin{itemize}
    \item Psi/chi functions
    \item Finite dilution
    \item Temperature effects
  \end{itemize}

\item Monte Carlo
  \begin{itemize}
    \item
  \end{itemize}
  
\item Heterogeneous Self-shielding 
  \begin{itemize}
    \item Equivalence theory
    \item Escape probabilities
    \item Collision probabilities
    \item Rational models
    \item Dancoff factors
    \item 1/E fluxes
  \end{itemize}

\item Pin-cell code: 
  \begin{itemize}
    \item Flux disadvantage factors
    \item Fine structure effects
    \item Coefficients
  \end{itemize}
\end{enumerate}


\clearpage
\topic{Exam 2 Re-cap}
Expect item 1 \& 2 in final and oral. 
\begin{enumerate}
\item $\kinf$ 
  \begin{itemize}
  \item One-group
  \item 
  \end{itemize}

\item $\keff$


\item Multi-group approximation: know how to derive the following two. 

\item Analytical solution of one group diffusion equation: 

\item Two group diffusion equations: very simple set-up,

\item Numerical solution of 1D diffusion equations\footnote{The most important topic in solving diffusion equations, especially the iterative methods}: 

\item Fission product equations

\item Nuclide depletion equations: it is exactly the same as the FP equations. 

\item Coupled first-order ODEs, matrix solution, look hard to know how to use IF to do 1st order ODE. 
\end{enumerate}


\clearpage
\topic{Exam 3 Re-cap}
\begin{enumerate}
\item Point kinetics equations


\item Point kinetics equations: 
  \begin{itemize}
    \item Prompt-jump
    \item In-hour equation
    \item Asymptotic periods
  \end{itemize}

\item Reactivity units

\item Inverse kinetics equations: 1st order ODE again. 

\item Fuchs-Nordheim model for RIAs: nice for intuition
  \begin{itemize}
    \item Deposited energy
    \item Doppler coefficients
  \end{itemize}

\item Transient feedback effects

\item Nodal methods\footnote{weighted residual is not on the final}

\item Transverse-integration procedure

\item Homogenization methods \footnote{discontinuity factors are very important!}

\item Pin power reconstruction

\item Perturbation theory: 

\item Adjoint fluxes: if we put a neutron at this place this energy, what is the probability it would cause fission. 

\item Adjoint fluxes in kinetics parameters\footnote{In operating reactors, they measure the control rod worth; they are no better than the delayed spectra}
  \begin{itemize}
  \item Delayed spectra
  \item $\bar{I}$
  \end{itemize}

\item Loading pattern optimization
\end{enumerate}
%%%%%%%%%%%%%%%%%%%%%%%%% Final Exam End %%%%%%%%%%%%%%%%%%%%%%%%%%%%




\end{document}

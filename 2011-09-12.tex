\documentclass{school-22.101-notes}
\date{September 12, 2011}

\begin{document}
\maketitle


\topic{Four Quantum Mechanics Postulates}
\subtopic{Postulate 1} 
\begin{axiom} There are two versions:
  \begin{subaxiom}
    The permissible values of the observables are the eigenvalues from solving the associated mathematical operators.
  \end{subaxiom}
  \begin{subaxiom}
    The properties of a quantum system are completely defined by specification of its state vector $\ket{\psi}$. The state vector is an element of a complex Hilbert space $H$ called the space of states. 
  \end{subaxiom}
\end{axiom}

We are going to explain this postulate through some specific examples:
\begin{enumerate}
\item \uline{Operators Definitions in 1D Free Particle Problem: } Position operators by definition are: $\hat{x}, \hat{y}, \hat{z},$ or as a vector $\hat{\underline{x}}$. Momentum operators by definition are:
\eqn{ \hat{p}_x &= - i \hbar \ppx,  &\hat{p}_y &= - i \hbar \ppy, &\hat{p}_z &= - i \hbar \ppz, & \Aboxed{\hat{p} &=  -i \hbar \gradient} }

Classically, energy is: $E = KE + PE = \frac{1}{2m} (p_x^2 + p_y^2 + p_z^2) + V(x,y,z)$. Quantum mechanically, the energy operators are:
\eqn{ \hat{H} = \frac{1}{2m} \left( \hat{p}_x^2 + \hat{p}_y^2 + \hat{p}_z^2 \right) + V(\hat{x}, \hat{y}, \hat{z}) = \boxed{ - \frac{\hbar^2}{2m} \laplacian + V(\hat{x}, \hat{y}, \hat{z})} }

\item \uline{Momentum eigenvalue problem for 1D free particle: }
\begin{align}
\hat{P} u_n(\uline{x}) &= p_n u_n(\uline{x}) & -i\hbar \ppx u_n(\uline{x}) &= p_n u_n(\uline{x}) \\
\ppx u_n &= i \frac{p_n}{\hbar} u_n   & u_n &= A e^{i \frac{p_n}{\hbar} x} = A e^{ikx} = A (\cos kx + i \sin kx) 
\end{align}
in which $\lambda = \frac{\hbar}{p} = \frac{2\pi}{k}$, and $k$ is the wave number defined by $k = \frac{p_n}{\hbar}$.

\textcolor{blue}{The eigenfunction of the momentum eigenvalue problem gives us a de Broglie wave traveling to the right hand, and the momentum value, or the eigenvalue of the problem is $p_n = \hbar k$.} Note this is a continuous solution, meaning there is no restriction on wavelength $\lambda$ or wave number k, because this is a 1D free particle problem. 

Is $\cos kx$ a wave function satisfying the momentum eigenvalue problem? We can consider:
\eqn{ \cos kx = \frac{1}{2} (e^{ikx} + e^{-ikx})  }
This is to say the eigenfunction $\cos kx$ is actually the combination of a wave traveling to the right $e^{ikx}$ and a wave traveling to the left $e^{-ikx}$, and it is not the solution of the momentum eigenvalue problem, hence it does not satisfy the momentum eigenvalue problem. 


\item \uline{Position eigenvalue problem for 1D free particle: }
The position eigenvalue problem looks like:
\eqn{ \hat{x} V_j (x) = x_j V_j (x) }
The solution to this problem is a special function Delta function $\delta (x - x_j)$, which at position $x_j$, $V_j(x)$ has a height of infinity, a width of 0, and an area of 1. Hence we get:
\begin{align}
\int_{-\infty}^{\infty} \dV f(x) \delta(x-x_j) &=  \int_{-\infty}^{\infty} \dV x \delta(x-x_j) = x_j = f(x) \\
\Rightarrow \int_{-\infty}^{\infty} \dV \delta(x-x_j) &= 1 
\end{align}


\item \uline{Energy eigenvalue problem for 1D free particle: }
Recall that energy operator is defined as a Hamiltonian Operator: $\hat{H}  = -\frac{\hbar^2}{2m} \gradient^2 + V(x,y,z)$. Because we are dealing with free particles, we can drop the potential term. The energy eigenvalue problem we are solving is then: 
\begin{align}
 \hat{H} W_n (\uline{x}) &= E_n W_n (\uline{x} ) \\
 -\frac{\hbar^2}{2m} \gradient^2 W_n (\uline{x}) &= E_n W_n (\uline{x} ) \\
 \ppxn2 W_n + \frac{2m E_n}{\hbar^2} W_n &= 0 \\
 \ppxn2 W_n + k^2 W_n &= 0  \\
 W_n &= A \sin kx + B \cos kx = A^{\prime} e^{ikx} + B^{\prime} e^{-ikx}
\end{align}
in which we replace $\frac{2m E_n}{\hbar^2}$ with $k_n^2$ because:
\eqn{ p = \hbar k \Rightarrow k = \frac{p}{\hbar} = \frac{\sqrt{2 m E}}{\hbar} }
Again keep in mind that for free particles there is no restriction on the k value, position or anything else. 


\item \uline{Summary: }
\begin{table}
\begin{tabular}{|c|c|c|c|} \hline
Observables & Operators & Eigenvalues & Eigenfunctions \\ \hline
Position & $\hat{x} = x$ & $x_j$ (no constrain) & $\delta(x-x_j)$. \\ \hline
Momentum & $\hat{p} = -i\hbar \ppx = - i \hbar \gradient $ & $p_n = \hbar k$ (no constrain) & $u_n = A e^{i \frac{p}{\hbar} x } = A e^{i k x} $ \\ \hline
Energy & $\hat{H} = \frac{\hat{p}^2}{2m} + V(\hat{x}) $ & $E_n = \frac{\hbar^2 k^2}{2m}$ (no constrain) & $W_n = A \sin kx + B \cos kx = A^{\prime} e^{ikx} + B^{\prime} e^{-ikx} $\\ \hline
\end{tabular}
\caption{Properties of Position, Momentum, Energy Operators}
\end{table}

if a function is the eigenfunction of the momentum operator, it must also be the eigenfunction of the energy operator. Proof:
\begin{align}
\hat{p} W_n &= \hbar k W_n \\
\hat{H} W_n &= \frac{\hat{p}^2}{2m} W_n 
= \frac{\hat{p}}{2m} (\hat{p} W_n) 
= \frac{\hat{p}}{2m} \hbar k W_n = \frac{\hbar k}{2m} \hat{p} W_n = \frac{\hbar^2 k^2}{2m} W_n = E_n W_n
\end{align}
Thus for any $W_n$ that is the eigenfunction of $\hat{p}$, it must also be the eigenfunction of $\hat{H}$.

This is the end of the first postulate, in which we relate physical observables (position, momentum, energy) with their mathematical operator, through whom we can solve for the possible values for the observables. 
\end{enumerate}

\end{document}

\documentclass{school-22.101-notes}
\date{October 12, 2011}

\begin{document}
\maketitle

%%%%%%%%%%%%%%%%%%%%%%%%%%% Angular Momentum %%%%%%%%%%%%%%%%%%%%
\lecture{Angular Momentum}
Angular momentum is a conserved quantity in a central force field. While a general potential is dependent on the spatial coordinate:
\eqn{ V(\uline{x}) = V(r,\theta, \phi)   }
a special potential that is only depend on radius (no $\theta, \phi$ dependency) is called a central force field because of its rotational symmetry. 

\topic{Physical Meaning of Momentums}
\textbf{Momentum is a generator of translation.}\footnote{This lecture is given on Oct. 1, 2012 by Prof. Ju Li.} We consider an operator $T$ that moves a $\psi(x)$ to $\psi(x-\Delta x)$: 
\eqn{ T \psi(x) = \psi(x-\Delta x)} 
We perform the Taylor expansion on $\psi(x - \Delta x)$: 
\begin{align}
  \psi(x - \Delta x) &= \psi(x) + (-\Delta x) \ppsipx + \frac{(-\Delta x)^2}{2!} \ppsipxn2 + \frac{(-\Delta x)^3}{3!} \ppsipxn3 + \cdots \\
  &= \left[ 1 - \Delta x \ppx + \frac{(-\Delta x \ppxn2)^2}{2!} + \frac{(-\Delta x \ppxn3)^3}{3!} + \cdots \right] \psi(x) 
\end{align}
Using $e^x = 1 + x + \frac{x^2}{2!} + \frac{x^3}{3!} + \cdots$, we can write, 
\begin{align}
  \psi(x - \Delta x) &= \exp \left( -\Delta x \ppx \right) \psi(x) \\
  &= \exp \left(- \frac{i \Delta x (-i \hbar \ppx)}{\hbar} \right) \psi(x) \\
  &= \exp \left(-\frac{i \Delta x \cdot \hat{P} }{\hbar} \right) \psi(x)
\end{align}
That is, the position transform vector is the momentum operator. 


\textbf{Angular momentum is a generator of rotation.} We consider an operator $R(n, \theta)$ that rotate a $\psi$ to $\psi_{\mathrm{rotational}}$: 
\eqn{ R(\vec{n}, \theta) \psi = \psi_{\mathrm{rotational}} }
We can prove that 
\eqn{ R(\vec{n}, \theta) = \exp \left( - \frac{i \theta}{\hbar} \vec{n} \cdot \vec{J} \right) }
So we want to prove, 
\begin{align}
\bra{\psi_{\mathrm{rot}}} V \ket{\psi_{\mathrm{rot}}} &= R(\vec{n}, \theta) \bra{\psi} V \ket{\psi} \\
\bra{\psi} R^+ V R \ket{\psi} &= \bra{\psi} \exp \left( \frac{i \theta \vec{n} \cdot \vec{J} }{\hbar} \right) V \exp \left( - \frac{i \theta \vec{n} \cdot \vec{J} }{\hbar} \right) \ket{\psi} 
\end{align}
where we can expand: 
\begin{align}
V + \frac{i \theta}{\hbar} [ \vec{n} \cdot \vec{J}, \vec{V} ] &= \left( 1 + \frac{i \theta n J}{\hbar} \right) V \left(  1 - \frac{i \theta n J}{\hbar} \right) \\
&= V + \frac{i \theta V}{\hbar} - \frac{i \theta V}{\hbar}  
\end{align}


\topic{Mathematical Derivation of Angular Momentum Commutators}
First we introduce the Levi-Civita symbols. 
\begin{itemize}
\item Index rule: 
  \eqn{ w_i = \epsilon_{ijk} n_j V_k = \Sum_j \Sum_k \epsilon_{ijk} n_j V_k }
\item Changing index: 
  \eqn{ \epsilon_{ijk} = - \epsilon_{jik} = -\epsilon_{ikj} }
  \eqn{ \epsilon_{ijk} \epsilon_{ij'k'} = \delta_{jj'} \delta_{kk'} - \delta_{jk'} \delta_{j'k} }
\item Final results: other terms are 0, 
\eqn{ \epsilon_{123} &= \epsilon_{231} = \epsilon_{312} = 1 & \epsilon_{213} &= \epsilon_{312} = \epsilon_{132} = -1 }
\end{itemize}


After simplification, we can show that, 
\begin{align}
\bra{\psi} \frac{i}{\hbar} [n_j J_j, V] \ket{\psi}  &= \vec{n} \cross \bra{\psi} \vec{V} \ket{\psi} \\
\bra{\psi} \frac{i}{\hbar} [n_j J_j, V_i] \ket{\psi} &= \epsilon_{i,j,k} n_j \bra{\psi} V_k \ket{\psi} \\
\frac{i}{\hbar} [J_j, V_i] &= \epsilon_{ijk} V_k \\
[V_i, J_j] &= i \hbar \epsilon_{ijk} V_k \\
[\vec{x}, \vec{P} ] &= i \hbar \epsilon_{ijk} V_k \\
\vec{J} &= \vec{L} = \vec{x} \cross \vec{P} 
\end{align}


Application of commutators:
\begin{enumerate}
\item 
\eqn{ J^2 &= \vec{J} \cdot \vec{J} = J_x^2 + J_y^2 + J_z^2 = J_i J_i }
\item Using $[AB,C] = A[B,C] + [A,C] B$, 
\eqn{ J^2 &= [J^2, J_3] &= [J_i J_i, J_3] = J_i [J_i, J_3] + [J_i, J_3] J_i }
\item Further simplify, 
\eqn{ J^2 &= J_i (i \hbar \epsilon_{i3k} J_k) + (i\hbar \epsilon_{i3k} J_k) J_i  = i\hbar \epsilon_{i3k} (J_i J_k + J_k J_i) = 0 }
\item Basically we've shown that for any $n_x, n_y, n_z$, 
\eqn{ [J^2, \vecn \cdot \vec{J} ] = [J^2, n_x J_x + n_y J_y + n_z J_z] = 0}
\item If we define the raising operator $J_+ = J_x + iJ_y$, and the lowering operator $J_- = J_x - iJ_y$. Using last rule, we know immediately that 
\eqn{ [J^2, J_+] = [J^2, J_-] = 0} 
\item Now we show how $J_+, J_-$ relate to $J_z$: 
  \begin{enumerate}
  \item To start, 
    \begin{align}
      [J_-, J_z] &= [J_x - i J_y, J_z]  \\
      &= \left( \epsilon_{132} J_y - i \epsilon_{231} J_x \right) i \hbar \\
      &= \hbar (J_x - i J_y)  = \hbar J_- 
    \end{align}
  \item We add an arbitrary basis $\ket{z}$ to both side and flip the order inside of the commutator, 
    \eqn{ [J_z, J_-] \ket{z} &= - \hbar J_- \ket{z} }
    We further simplify the LHS: 
    \eqn{ J_z J_- \ket{z} - J_- J_z \ket{z} = J_z J_- \ket{z} - z \hbar \ket{z} }
    Hence, 
    \eqn{ J_z J_- \ket{z} = z \hbar J_- \ket{z} - \hbar J_- \ket{z} = (z-1) \hbar J_- \ket{z} \label{loweringop} } 
  \item Keep in mind that \uline{by definition}, 
    \eqn{ J_z \ket{z} = z \hbar \ket{z} }
    In a similar fashion for some unknown $\beta$ we can write 
    \eqn{ J^2 \ket{z} = \beta \ket{z} }
  \item Eq.~\ref{loweringop} gives us a sense that the lowering operator stays in the same `family', though its eigenvalue is one less. 
    \begin{align}
      J_- \ket{z} &= \alpha \hbar \ket{z-1} \\
      \bra{z} J_+ J_- \ket{z} &= |\alpha|^2 \hbar^2 \\
      \mathrm{LHS} &= (J_x + i J_y) (J_x - iJ_y) = J_x^2 + i (J_y J_x - J_x J_y) + J_y^2 \\
      &= J_x^2 + J_y^2 + \hbar J_z = J^2 - J_z^2 + \hbar J_z \\
      \bra{z} J^2 - J_z^2 + J_z \hbar \ket{z} &= |\alpha|^2 \hbar^2 \\
      \bra{z} J^2 \ket{z} - z^2 \hbar^2 + z^2 \hbar^2 &= |\alpha|^2 \hbar^2 \\
      \bra{z} J^2 \ket{z} &= |\alpha|^2 \hbar^2 
    \end{align}
    where the $\bra{z} J^2 \ket{z}$ term is kind the $\beta \hbar^2$ term that keeps $J^2$ in the same family. 
  \item We have to worry about we cannot just infinitely lower $z$. That is, the $\beta$ term has to satisfy the following three relations: 
    \begin{align}
      \left\{ \begin{array}{c} 
        z_{\max} - z_{\min} = \mathrm{integer}  \\
        \beta - z_{\min}^2 + z_{\min} = 0 \\
        \beta - z_{\max}^2 - z_{\max} = 0
      \end{array}
      \right. 
    \end{align}
    The only solution is, 
    \eqn{ z_{\min} = - z_{\max} = \left\{ \begin{array}{cc} \mathrm{integer} & \mathrm{Bosons} \\ \mbox{half integer} & \mathrm{Fermions}  \end{array} \right.   }
    Then we can write 
    \eqn{ |\alpha| = \sqrt{j(j+1) - m(m-1)} }
  \item If we call $z_{\max} = j$, and make up a $m = z \in [-j, j]$, then we have 
    \begin{align}
      \Aboxed{J^2 \ket{j m} &= j(j+1) \hbar^2 \ket{j m}} \\
      \Aboxed{J_z \ket{j m} &= m\hbar \ket{j m} } \\
      \Aboxed{ J_{\pm} \ket{j m} &= \sqrt{j(j+1) - m(m\pm 1)} \hbar \ket{jm\pm 1} }
    \end{align}

  \item Now if we consider a half-spin particle (neutrons, protons, electrons), that is $j = 1/2$. We can use the matrix notation, where bra becomes row vector, ket becomes column vector, and scalar becomes matrix. We can re-write the ket basis into 
    \eqn{ \ket{\frac{1}{2}} &= \left( \begin{array}{c} 1 \\ 0 \end{array} \right) & \ket{-\frac{1}{2}} &= \left( \begin{array}{c} 0 \\ 1 \end{array} \right) }
    Then we can solve the $\frac{\hbar}{2} \ket{\frac{1}{2}} = J_z \ket{\frac{1}{2}}, - \frac{\hbar}{2} \ket{-\frac{1}{2}} = J_z \ket{-\frac{1}{2}}$ equations for the-now-matrix-$J_z$: 
    \begin{align}
      \frac{\hbar}{2}  \left( \begin{array}{c} 1 \\ 0 \end{array} \right) &=  \left( \begin{array}{cc} a & c \\ b & d   \end{array} \right)  \left( \begin{array}{c} 1 \\ 0 \end{array} \right) = \begin{array}{c} a \\ b \end{array} \\
      - \frac{\hbar}{2}  \left( \begin{array}{c} 0 \\ 1 \end{array} \right) &=  \left( \begin{array}{cc} a & c \\ b & d   \end{array} \right)  \left( \begin{array}{c} 0 \\ 1 \end{array} \right) = \begin{array}{c} c \\ d \end{array} \\
      J_z &= \frac{\hbar}{2} \left( \begin{array}{cc} 1 & 0 \\ 0 & -1 \end{array} \right) 
    \end{align}
    Similarly we can solve for the matrix representation of $J_+, J_-$ as well, 
    \begin{align}
      J_+ \ket{ \frac{1}{2} } &= 0  \\
      \left( \begin{array}{c} a \\ b \end{array} \right) &= 0 \\
      J_+ \ket{ - \frac{1}{2} } &= \hbar \sqrt{-j(j+1) - m(m+1)} \ket{\frac{1}{2} }  \\
      &= \hbar \sqrt{ \frac{1}{2} \frac{3}{2} - \left( - \frac{1}{2} \right) \left( - \frac{1}{2} + 1 \right) } \ket{ \frac{1}{2} } \\
      &= \ket{ \frac{1}{2} } \\
      \Aboxed{J_+ &= \hbar \left( \begin{array}{cc} 0 & 1 \\ 0 & 0 \end{array} \right)}
    \end{align}
    We know that $J_-$ is just the transpose of $J_+$, hence, 
    \eqn{ \Aboxed{J_- &= \hbar \left( \begin{array}{cc} 0 & 0 \\ 1 & 0 \end{array} \right)}}
    Knowing what $J_+, J_-$ are, and knowning that $J_+ = J_x + i J_y, J_- = J_x - iJ_y$, and knowing that $J_i = \frac{\hbar}{2} \sigma_i$ where $\sigma_i$ is the Pauli matrix obeying $[ \sigma_i, \sigma_j] = 2 \epsilon_{ijk} \sigma_k$, we can write $J_x, J_y, J_z$: 
    \begin{align}
      J_x &= \frac{\hbar}{2} \left( \begin{array}{cc} 0 & 1 \\ 1 & 0 \end{array} \right) \\
      J_y &= \frac{\hbar}{2} \left( \begin{array}{cc} 0 & -i \\ i & 0 \end{array} \right) \\
      J_z &= \frac{\hbar}{2} \left( \begin{array}{cc} 1 & 0 \\ 0 & -1 \end{array} \right) 
    \end{align}

  \item Comment: We typically draw spins on a sphere. The reason is,  for instance if we make a z measurement and get $m = j$ (we could get $m = j-1$ etc), 
    \begin{itemize}
      \item Sphere, because $J^2 = j(j+1) \hbar^2$. The basis/direction is $\ket{m =j}$. 
      \item The circular motion is from that, if we make a z measurement, there is still uncertainty in the x and y term. That is, we know the average x measurement should be zero from 
        \eqn{ \bra{m=j} J_x \ket{m=j}  = 0} 
        But the fluctuation in $x,y$ are not zero, as we don't know exactly what they are, only the sum of their products: 
        \eqn{ \expect{J_x^2} + \expect{J_y^2} + (j \hbar)^2 = j^2 \hbar^2 + j^2 \hbar^2 }
    \end{itemize}


\end{enumerate}
\end{enumerate}




%%%%%%%%%%%%%%%% %%%%%%%%%%%%%%%%%%%%%%%%%%%%%%%%%%%
\topic{Orbital Angular Momentum}
This is similar to its classical description of angular momentum. 

\subtopic{Construct Cartesian Components of $\hat{L}$}
We define the orbital angular momentum like:
\begin{align}
\hat{\uline{L}} &= \hat{\uline{r}} \cross \hat{\uline{p}} \\
\hat{\uline{L}}_x &= \hat{y} \hat{p}_z - \hat{z} \hat{p}_y = - i\hbar \left( y \ppz - z \ppy \right) \\
\hat{\uline{L}}_y &= \hat{z} \hat{p}_x - \hat{z} \hat{p}_x = - i\hbar \left( z \ppx - x \ppz \right) \\
\hat{\uline{L}}_z &= \hat{x} \hat{p}_y - \hat{y} \hat{p}_x = - i\hbar \left( x \ppy - y \ppx \right) 
\end{align}

\subtopic{Check the Components We Just Constructed}
The question we want to ask is, `can we find the eigenvalues of each of the operators simultaneously?'

To answer the above question, we consider whether the components commute with each other. Because if $\left[ \hat{L}_x, \hat{L}_y \right] = 0$, that is to say we can pin $L_x, L_y$ at the same time. 

Though when we derive it for real (see notes), they come out to be:
\eqn{ \left[ \hat{L}_x, \hat{L}_y \right] = i \hbar ( \hat{x} \hat{P}_y - \hat{y} \hat{P}_x) = i\hbar \hat{L}_z }
\eqn{ \left[ \hat{L}_y, \hat{L}_z \right] = i \hbar ( \hat{y} \hat{P}_z - \hat{z} \hat{P}_y) = i\hbar \hat{L}_x }
\eqn{ \left[ \hat{L}_z, \hat{L}_x \right] = i \hbar ( \hat{z} \hat{P}_x - \hat{x} \hat{P}_z) = i\hbar \hat{L}_y }

That is to say, $\hat{L}_x, \hat{L}_y, \hat{L}_z$ do not have common eigenstates, and that we cannot find $L_x, L_y, L_z$ simultaneously. 

\subtopic{Define the Orbital Angular Momentum Operator $\Lhat^2$}
What the above argument leads to is the $\hat{L}^2$,  a more physical term, and the square root of its eigenvalue is the angular momentum. 
\eqn{ \left[ \hat{L}^2, \hat{L}_x   \right] =  \left[ \hat{L}^2, \hat{L}_y   \right] =  \left[ \hat{L}^2, \hat{L}_z   \right] = 0   }
We can solve for the total angular momentum and the azimuthal component simultaneously in a coordinate system as illustrated in Figure~\ref{3DCS}. 
\begin{figure}
    \centering
    \includegraphics[width=2in]{images/qm/3DCS.png}
    \caption{Set-up of the 3D Coordinate System\label{3DCS}}
\end{figure}
\eqn{ \hat{L}_z = - i \hbar \ppphi }
\eqn{ \hat{L}^2 =  - \hbar^2 \left[ \frac{1}{\sin \theta} \pptheta \left( \sin \theta \pptheta \right) + \frac{1}{\sin^2 \theta} \ppphin2  \right] }

\subtopic{Define the Total Angular Momentum $\Jhat$}
We define the Total Angular Momentum as:
\eqn{ \Jhat = \Lhat + \Shat }
Notice we cannot solve for the three components of $\Jhat$ simultaneously neither, because: 
\eqn{ \left[ \Jhat_x, \Jhat_y \right] = i \hbar \Jhat_z,  \left[ \Jhat_y, \Jhat_z \right] = i \hbar \Jhat_x, \left[ \Jhat_z, \Jhat_x \right] = i \hbar \Jhat_y }


\subtopic{Eigenvalues of $\Lhat, \Jhat$}
See Liboff 9.3 for how we get the following equations: 
\eqn{ \boxed{ \hat{L}^2 \phi_{l,m} (\theta, \phi) = \hbar^2 l (l+1) \phi_{l,m} (\theta, \phi ), \fsp \fsp \hat{L}_z \phi_{l,m} (\theta, \phi) = \hbar m \phi_{l,m} (\theta, \phi ) } }
in which $l = 0,1,2, \cdots, m = -l, -l+1, \cdots, 0, \cdots l$. If we solve for $\hat{L}_x$ or $\hat{L}_y$ instead we will reach the same results as above. 

Similarly $\Jhat$ would give us:
\eqn{  \Jhat^2 \phi_{l,m} (\theta, \phi) = \hbar^2 j (j+1) \phi_{l,m} (\theta, \phi ), \fsp \fsp \Jhat_z \phi_{l,m} (\theta, \phi) = \hbar m_j \phi_{l,m} (\theta, \phi )  }
in which $j = 0,1/2,1,3/2,2, \cdots, m = -j, \cdots, j$. If we solve for $\Jhat_x$ or $\Jhat_y$ instead we will reach the same results as above. 

\subtopic{Eigenfunctions $\phi_{l,m}$ is Spherical Harmonics $Y_l^m(\theta, \phi)$}
The eigenfunction in the above equation, $\phi_{l,m}$, are commonly called the \textbf{Spherical Harmonics $Y_l^m (\theta, \phi)$}:
\eqn{ Y_l^m (\theta, \phi) = \frac{1}{\sqrt{2 \pi}} e^{im\phi} P_l^m (\phi)   }
A couple of things about the Spherical Harmonics:
\begin{itemize}
\item $\psi( r, \theta, \phi) = R(r) Y_{lm} (\theta, \phi)$. 
\item $\int_0^{2\pi} \dphi \int_0^{\pi} \sin \theta \dtheta |Y_{lm} (\theta, \phi) |^2 = 1.$
\item Spherical Harmonics also implies that the unit of angular momentum is $\hbar$.
\item Degeneracy: \textbf{For a given $l$ value, there are $2l+1$ number of degeneracy.} For instance, for $l=5$, we have 11 eigenstates $Y_5^5, Y_5^4, \cdots Y_5^{-5}$ that correspond to the same $l$, or the same $L^2 = 30 \hbar^2$. That is, for the same $L^2$, the projection onto the azimuthal plan are degenerate.  
\item $L_z$ is not continuous. Given a $L^2$, we can draw a bunch of cones, each surface describe the superposition of possible eigenfunctions. L will never entirely align with the z axis. 
\end{itemize}


\subtopic{Summary of Quantum Numbers}
\begin{enumerate}
\item Orbital Angular Momentum Quantum Number $l$: integers, $0 \le l \le n$. A subset of j is the solution from orbital component. 
\item Total Angular Momentum Quantum Number $j$: integer steps, $|l-s|, \cdots, |l+s|$. It completely construct $l$ and $s$. 
\item $m_j$: $-j, \cdots j$ (includes 0 when j is an integer, and does not include 0 when j is a half-integer). It's degeneracy is $2j+1$. 
\end{enumerate}

\begin{table}[h!]
\begin{tabular}{|p{1.5in}|p{1.5in}|p{1.5in}|p{1.5in}|} \hline
 & $\Lhat$ & $\Shat$ &$\Jhat = \Lhat + \Shat$ \\ \hline
\multirow{3}{*}{Commutating Relations} &
   $\left[ \Lhat_x, \Lhat_y \right] = i \hbar \Lhat_z$ &  $\left[ \Shat_x, \Shat_y \right] = i \hbar \Shat_z$ &  $\left[ \Jhat_x, \Jhat_y \right] = i \hbar \Jhat_z$ \\
&  $\left[ \Lhat_y, \Lhat_z \right] = i \hbar \Lhat_x$ &  $\left[ \Shat_y, \Shat_z \right] = i \hbar \Shat_x$ &  $\left[ \Jhat_y, \Jhat_z \right] = i \hbar \Jhat_x$ \\
&  $\left[ \Lhat_z, \Lhat_x \right] = i \hbar \Lhat_y$ &  $\left[ \Shat_z, \Shat_x \right] = i \hbar \Shat_y$ &  $\left[ \Jhat_z, \Jhat_x \right] = i \hbar \Jhat_y$ \\ \hline
Quantum Numbers & $ l = 0,1,2,\cdots $ & $s = 0, 1/2, 1, \cdots$               & $ j =|l-s|,\cdots, |l+s|$ \\
(all in integer steps) & $-l \le m_l \le l  $  & $ -s \le m_s \le s$  & $-j \le m_j \le j$  \\ \hline
\multirow{2}{*}{Eigenvalues} 
& $\expect{L^2} = \hbar^2 l (l+1)$ 
& $\expect{S^2} = \hbar^2 s (s+1)$
& $\expect{J^2} = \hbar^2 j(j+1)$\\ 
& $\expect{L_z} = \hbar m_l$ 
& $\expect{S_z} = \hbar m_s$ 
& $\expect{J_z} =  \hbar m_j$  \\ \hline
\end{tabular}
\caption{Comparison of Quantum Numbers $l,s,j$}
\label{quantum-numbers}
\end{table}


\topic{Additions of Angular Momentum}
Examples:
\begin{enumerate}
\item $2e^-: \Lhat = \Lhat_1 + \Lhat_2 $ 
\item $ 1 e^-: \Jhat = \Lhat + \Shat$
\item \ce{^2H} = p+n: l =0. $\Jhat = \Shat_p + \Shat_n$. 
\end{enumerate}

\uline{Example 1: Adding Two Orbital Momentum:} we want to find the $(l,m)$ associated with $\Lhat= \Lhat_1 + \Lhat_2$ from the four quantum numbers $l_1, m_1, l_2, m_2$. 
\begin{itemize}
\item $\Lhat^2 = (\Lhat_1 + \Lhat_2)^2 $
\item $ \Lhat_z = \Lhat_{1z} + \Lhat_{2z} \Rightarrow m_{\mathrm{max}} = m_{\mathrm{1,max}} + m_{\mathrm{2,max}} = l_1 + l_2.$
$\fsp l_{\mathrm{max}} = m_{\mathrm{max}} = l_1 + l_2.$
\item $ \mbox{Total \# of independent states } N = (2 l_1 + 1) (2l_2 + 1) = \Sum_{l_{\mathrm{min}}}^{l_{\mathrm{max}}} (2l+1).$
$\fsp\Rightarrow l_{\mathrm{min}} = |l_2 - l_1 |.$ That is,
\eqn{ l = |l_2 - l_1|, \cdots , l_1 + l_2}
\end{itemize}

\uline{Example 2: Adding Two Electrons}

Given: $ 2e^-$, 1 $e^-$ \@ $l_1 = 1$, 1 $e^-$ \@ $l_2 = 2$. Answer: 
\begin{itemize}
\item $l = 1,2,3$.
\item $L = \hbar \sqrt{l (l+1)} = \hbar \sqrt{2}, \hbar \sqrt{6}, \hbar \sqrt{12}.$.
\item $N = \Sum_1^3 (2l+1) = 3 + 5 + 7 = 15$, or $N = 3 \times 5 = 15$. 
\end{itemize}





\end{document}

\documentclass{school-22.211-notes}
\date{April  2, 2012}

\begin{document}
\maketitle

\lecture{Numerical Solutions for Diffusion Theory}
\topic{Simple Numerical Methods to Solve Diffusion Equations}
\subtopic{Derivation of Neutron Balance Equation}


\subtopic{Derivation of Interface Flux}


\subtopic{Derivation of Finite Difference Equations}


\subtopic{Derivation of Boundary Condition}
Interior of the geometry, the boundary conditions are implied. The two outter boundaries are the 


If $D \to 0$, we get zero flux boundary condition; if $D \to \infty$, we get zero current boundary condition. 

\topic{Matrix Representation of 1D Slab Diffusion Equations}
\subtopic{Construct Matrix}
If we use $L$ for the $n-1$ term, $U$ for the $n+1$ term, $D$ for the $n$ term, $T$ for the transport term, we can express the finite difference equation in matrix form as: 
\begin{align}
xx
\end{align}

Block matrix: there is no coupling of group 1 left flux to the group 2 right flux. 

\subtopic{Method 1: Sequential Source/Fixed-Source Problem}
\begin{align}
[A] [\phi] &= [M] [\phi] + [S] \\
[A - M ] [\phi] &= [S] \\
[\phi] &= [A - M]^{-1} [S]
\end{align}
In this method, we guess $\keff$, and given a source we can solve for the flux. We never solve real questions this way, because 

If we have a super-critical problem, then the only way to get criticality with an additional source is through negative flux. 

A reactor is like an amplifier; the closer it is to criticality the more amplification it is; when critical, the amplification is infinity. 


\subtopic{Method 2: Direct Matlab Eigenvalue Solver}

Notice that the eigenvectors are arbitrarily normalized; hence it is fine if the flux is negative. 



\subtopic{Method 3: Power Iterations}



The rate our fission source converges depends on material, geometry etc. \hi{The dominance ratio} is defined as $\frac{\lambda_1}{\lambda_2}$; and it is 0.97 for this example. If we know the dominance ratio of the problem, we can estimate the number of iterations needed to converge. 


\subtopic{Method 3+: Power Iterations with Gauss-Jacobi Numerical Inversion of Flux Matrix}
In a steady-state problem, it does not matter whether we fully converge our flux iteration. 

\topic{HW5: Numerical Solution of Solving Two-Group Diffusion Problems}
It is not just iterative convergence that we care; spatial convergence is required as well. 

\topic{Summary}
Remember for real 3D problem,
\begin{itemize}
\item Matrix inversion if of order $N^3$, so in real applications no matrix inversion;
\item Finding all eigenvalues if at least order $N^2$;
\item Iterative inversion must be order $N$ to be practical for large problems;
\item Multi-level iteration is a practical necessity. 
\end{itemize}


\end{document}

\documentclass{school-22.101-notes}
\date{October 17, 2011}

\begin{document}
\maketitle


%%%%%%%%%%%%%%%%%% New Part: Nuclear Structure %%%%%%%%%%%%%%%%%%%%%%%%%%%%%%%%%
\lecture{Bound State of A Deuteron, Finite Square Well\label{2H-bound-state}}
Recall that we've solved the simple finite square well problem in Chapter~\ref{finite-square-well} (recall the even parity and odd parity cases). Now we consider the simplest bound state nucleus: a deuteron \ce{^2H}. We will show that a bound state problem can be reduced into three components:
\begin{itemize}
\item An eigenvalue problem (Section \ref{2H-eigenvalue});
\item An analytical relation and a quantitative relation between $V_0, \Gamma_0, E_0$ (Section \ref{2H-relation});
\item The spin dependence of the nuclear force (Section \ref{2H-spin}).
\end{itemize}
Reference: Liboff 10.5, Krane 4.1, 4.2, 4.4.

\topic{Reduce Bound State Problem into Eigenvalue Problem \label{2H-eigenvalue}} 
\begin{enumerate}
\item Write out the Hamiltonian: 
\eqn{ \Hhat = \frac{1}{2 m_n} \phat_n^2 + \frac{1}{2 m_p} \phat_p^2 + V_{\mathrm{nuc}} (\uline{x}_n - \uline{x}_p ) }
in which 
\begin{align}
 V_{\mathrm{nuc}} (\uline{x}_n - \uline{x}_p ) &= \mbox{Potential interaction (only depends on the radial distance b/w the two particles)} \\
 (\uline{x}_p, \phat_p; \uline{x}_n, \phat_n) &= \mbox{State vectors of the two particles} 
\end{align}
We assume central potential. 6-degrees of freedom = \# of coordinates in the state vector

\item Convert to Center-of-Mass frame: 
\begin{align}
 H \to H_{\mathrm{CM}} + H_{\mathrm{Rel}}, \fsp \fsp (\uline{x}_p, \uline{p}_p; \uline{x}_n, \uline{p}_n) \to (\uline{r}, \uline{p}_{\mathrm{Rel}}; \uline{R}, \uline{p}_{\mathrm{CM}} ) \\
\begin{dcases*} 
\uline{R} = \frac{1}{m_n + m_p} (\uline{x}_p x_p + \uline{x}_n x_n) & C.M. position\\
\uline{r} = \uline{x}_p - \uline{x}_n & Relative position \\
\uline{p}_{\mathrm{CM}} = \uline{p}_{n} + \uline{p}_p & Momentum of CM \\
\uline{p}_{\mathrm{Rel}} = \frac{m_p \uline{p}_n - m_n \uline{p}_p}{m_n + m_p} & Momentum of coordinate relative to C.M \\
\end{dcases*}
\end{align}
Then a deuteon Hamiltonian becomes:
\begin{align}
\Hhat_d &= \overbrace{\frac{1}{2 \underbrace{M}_{=m_n + m_p}} \uline{\Phat}_{\mathrm{CM}}^2}^{ = \hat{E}_{\mathrm{CM}} \to  0} + \frac{1}{2 \underbrace{\mu}_{=\frac{m_n m_p}{m_n + m_p}}} \uline{\Phat}_{\mathrm{Rel}}^2 + V_{\mathrm{nuc}} (\uline{r}) \\
&= \frac{1}{2 \mu} \uline{\Phat}_{\mathrm{Rel}}^2 + V_{\mathrm{nuc}} (\uline{r}) = -\frac{\hbar^2}{2 \mu} \gradient_r^2 + V_{\mathrm{nuc}} (\uline{r}) 
\end{align}
in which $\gradient_r^2$ is Laplacian in relative coordinates. Now we've reduced the problem into a `single-particle' problem that described the reduced mass $\mu$ about the center of mass in $\uline{r}$. 

\item A qualitative evaluation of $V_{\mathrm{nuc}} (\uline{r})$ \footnote{Krane, p.80}: 
    \begin{enumerate}
    \item At short distances: very strong, overcome the Coulomb repulsion of protons in the nucleus; 
    \item At long distances: very week and negligible; 
    \item Some particles, like $e^-$, are immute to $V_{\mathrm{nuc}}$;
    \item The nucleus potential looks like Figure~\ref{nuclear-potential-radius}. 
    \begin{figure}
        \centering
        \includegraphics[width=3in]{images/deuteron/nuclear-potential-radius.png}
        \caption{Deuteron Electromagnetic Nuclear Potential\label{nuclear-potential-radius}}
    \end{figure}
    \end{enumerate}

\item A quantitative evaluation of $\uline{\gradient}_r^2$: 
\begin{align}
- \frac{\hbar^2}{2 \mu} \uline{\gradient}_r^2 &= - \frac{\hbar^2}{2 \mu} \left[ \pprn2 + \frac{2}{r} \ppr \right] + \frac{1}{2 r^2 \mu} \overbrace{ (-\hbar^2) \left[ \frac{1}{\sin \theta} \pptheta \left( \sin \theta \pptheta \right) + \frac{1}{\sin^2 \theta} \ppphin2 \right]}^{\to \Lhat^2}  \\
&= - \frac{\hbar^2}{2 \mu} \left[ \pprn2 + \frac{2}{r} \ppr \right] + \frac{\Lhat^2}{2 r^2 \mu} 
\end{align}

\item Now we have our eigenvalue problem:
\eqn{ \left[ - \frac{\hbar^2}{2 \mu} \left( \pprn2 + \frac{2}{r} \ppr \right) + \frac{\Lhat^2}{2 r^2 \mu} + V_{\mathrm{nuc}} (\uline{r})  \right] \psi_n (r, \theta, \phi) = E_n \psi_n (r, \theta,\phi) }

\item To evaluate the above eigenvalue problem, notice we have
\eqn{ \left[ \Lhat^2, \Hhat \right]   = \left[ \Lhat_z , \Hhat \right] = 0 }
suggesting that $\Lhat^2, \Lhat_z, \Hhat$ shares the same eigenstate, which is the spherical harmonics $Y_l^m (\theta, \phi)$. Then we can write the wave function as: 
\eqn{ \psi_n (r, \theta, \phi) = \psi_{n,l} (r) Y_l^m (\theta, \phi) }
Also we know 
\eqn{ \Lhat^2 Y_l^m = \hbar^2 l(l+1) Y_l^m }
Plug the eigenvalues of $\Lhat^2$ back into the eigenvalue problem: 
\eqn{ \left[ - \frac{\hbar^2}{2 \mu}\left( \pprn2 + \frac{2}{r} \ppr \right)  + \frac{\hbar^2 l(l+1)}{2 \mu r^2 } + V_{\mathrm{nuc}} (\uline{r}) - E_{n,l}  \right] \psi_{n,l} (r) Y_l^m (\theta, \phi) = 0 }
Notice that the above equation is only dependent on r, so we can cross out the $Y_l^m (\theta, \phi)$ term, and call the radial component of the state function $r \psi_n (r) = u_n (r)$: 
\eqn{  \boxed{ \left[ - \frac{\hbar^2}{2 \mu}  \pprn2  + \underbrace{\frac{\hbar^2 l(l+1)}{2 \mu r^2 } + V_{\mathrm{nuc}} (\uline{r}) }_{\to V_{\mathrm{eff}} (\uline{r})} - E_{n,l}  \right] u_n (r)  = 0 } }
In a sense, we defined a modified potential term $V_{\mathrm{eff}} (r)$ that takes into account the angular momentum effect.  
\end{enumerate}

\clearpage
\begin{figure}
    \centering
    \includegraphics[width=5in]{images/deuteron/deutrium-V-eff.png}
    \\
    \includegraphics[width=5.5in]{images/deuteron/deutrium-V-eff-2.png}    
    \caption{$V_{\mathrm{eff}}$'s Effect on a Deutron}
    \label{V-eff}
\end{figure}
\textbf{Interpretations of Figure~\ref{V-eff}:}
\begin{enumerate}
\item Original $V_{\mathrm{nuc}} (r)$ is mostly attractive (negative); 
\item Angular momentum terms act as a repulsive barrier as it is positive;
\item $V_{\mathrm{eff}} (r)$ is still mostly attractive (negative), but weaker than $V_{\mathrm{nuc}} (r)$, In a word, \textcolor{blue}{$V_{\mathrm{eff}}$ lifts the potential up from the original $V_{\mathrm{nuc}}$, making the interaction weaker;} 
\item There are discrete bound states that fit into the potential. 
\end{enumerate}

Next: We've defined the particle potential. Now we are looking for the permitted energy levels $E_n$ that can fill in the potential. If we have a weaker potential, we can fit in less energy levels. Vice versa. In the next sections we will solve this problem for deutrons. 










\end{document}

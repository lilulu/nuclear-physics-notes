\documentclass{school-22.211-notes}
\date{February 13, 2012}

\begin{document}
\maketitle


%%%%%%%%%%%%%%%%%%%%%%%% Neutron Slowing Down and Thermalization %%%%%%%%%%%%%%%
\lecture{Neutron Slowing Down and Thermalization}

\topic{Summary: Reuss Ch 7 Neutron Slowing Down}
\begin{enumerate}
\item Decouple absorption and scattering is possible because absorption is complicated at lower energies, whereas scattering is the opposite (b/c inelastic and anisotropic aspects). 

\item Elastic vs. inelastic: elastic scattering has no threshold, making it the most important one in neutron slowing down. Inelastic scattering has a threshold of a few MeV for light nuclei, and a few tens of keV for heavy nuclei, making it mainly observed in the fuel materials particularly \ce{^{238} U}. 

\item In Lab system, laws of elastic collision: 
  \begin{align}
    \frac{E_{nf}}{E_{ni}} &= \frac{A^2 + 1 + 2A \cos \theta}{(A+1)^2} = \frac{1}{2} \left[ 1 + \alpha + (1-\alpha) \cos \theta \right] \\
    \cos \psi &= \frac{1 + A \cos \theta}{\sqrt{A^2 + 1 + 2A \cos \theta}} \\
    \alpha &= \frac{(A-1)^2}{(A+1)^2} = \mbox{min ratio between final n energy and initial} 
  \end{align}

\item In CMCS, scattering is isotropic in solid angle $\Omega$(except very high energy), which implies that $\cos \theta$ is uniform, and $E_{nf}$ is uniform:

  \begin{align}
    P(\theta) \dtheta &= \frac{1}{2} \sin \theta \dtheta = \frac{1}{2} \derivative |\cos \theta| \\
    P(E) \dE &= \frac{\dE}{(1-\alpha) E_i} 
  \end{align} 

\item In Lab system, scattering towards the front it favored. The mean of $\cos \psi$ is $\mu = \expect{ \cos \psi} = \frac{2}{3A}$. 

\item Lethargy, a unitless measurement of energy (the idea comes from laws of elastic collision governs an energy ratio):
  \eqn{ u = \ln \frac{E_{\mathrm{ref}}}{E} }
Notice as time goes, neutrons slow down, $u$ increases, making it like a measure of the age of the neutrons. Then we know that the uniform distribution of energy becomes a decreasing exponential distribution for lethargy gain: 
\begin{align}
w &= u_f - u_i = -\ln \left[ \frac{1}{2} ( 1 + \alpha + (1-\alpha) \cos \theta ) \right] \\
w_{min} &= 0 \\
w_{max} &= \epsilon = -\ln \alpha \\
P(w) \dw &= \frac{e^{-w}}{1 - \alpha} \dw \\
\expect{w} &= \xi = 1 - \frac{\alpha \epsilon}{1 - \alpha} 
\end{align}
$\xi$ is like the efficiency of slowing down by a nucleus. That is, neutrons advance by $\xi$ lethargy units on average at each collision. Then to overcome the total lethargy interval $U = \ln \frac{E_0}{E_1}$, the number of collisions needed is:
\eqn{ n = \frac{U}{\xi} }
\begin{figure}
  \centering
  \includegraphics[width=3in]{images/sl-d/slowing-down-parameter.png}
  \caption{Slowing Down parameters}
\end{figure}

\item Moderating power is the best measure of a material's ability to slow down neutrons. It has two forms:
\eqn{\mbox{per atom basis} = \xi \sigma_s \fsp \fsp \fsp \mbox{per volume basis} = \xi \Sigma_s   }
A good moderating material should have: high slowing down (hence light nuclei), low capture (D, Be, C), moderating power (take into account both high slowing down and high scattering xs). Water has the highest moderating power, but it requires an enriched fuel (around 1\%). 
\begin{figure}
  \centering
  \includegraphics[width=3in]{images/sl-d/moderator-comparison.png}
  \caption{Comparison of Characteristics of Moderators}
\end{figure}

\item Laws of inelastic collision. Elastic scattering is important for moderators because they are light; inelastic scattering is important for heavy materials like the fuel because they have almost no elastic collision. Minimum energy of the neutron for an inelastic collision is:
\eqn{ E_{\mathrm{threshold}} = \frac{A+1}{A} Q } 

\item Slowing down equations: a simplified Boltzmann equation performing neutron counts that depends on one variable, either velocity or energy or lethargy. We pick lethargy. Then we assume an infinite, homogeneous medium with a source uniform in space and time. 

\item First form of the slowing down equation (7.1.9):
  \begin{align}
    \rho(u) \du 
    &= \mbox{arrival density} = \begin{array}{l}
      \mbox{\# neutrons arriving per time and per volume} \\
      \mbox{in d$u$ between $u$ and $u + \du$ following a scattering to $u'$} 
      \end{array} \\
    &= \overbrace{\int_{-\infty}^u \Sigma_s (u') \Phi(u') \du'}^{\textcircled{1}} \overbrace{ P(u'\to u) \du }^{\textcircled{2}} \\
    \textcircled{1} &= \mbox{\# neutrons travelling in du' and scattered per time and per volume} \\
    \textcircled{2} &= \mbox{probability a neutron scattered at u' will be transferred in du} \\
    \rho(u) &= \int_{-\infty}^u \Sigma_s (u'\to u) \Phi(u') \du' \\
    S(u) + \rho(u) &= \boxed{S(u) + \int_{-\infty}^u \Sigma_s (u'\to u) \Phi(u') \du  = \Sigma(u) \Phi(u) }
  \end{align}
  In the specific case of COM isotropic, monatomic elastic slowing down, the transfer probability is:
  \eqn{P(u'\to u) = P(u\to u') = \frac{e^{-(u-u')}}{1-\alpha} }
\item Decay of neutron spectrum by successive scattering events (7.2.2): producing spectrum by adding up successive scattering events;
\item Slowing down without absorption (7.2.3). Asymptotic value of flux: 
  \eqn{ \Phi_{as} (u) = \frac{S}{\xi \Sigma_s(u)} }
  If we plot flux vs. lethargy, all isotopes would fluctuate (Placzek transient) before reaching the asymptotic flux value, except H which immediately reaches $\Phi_{as}$. 
\item Slowing down in hydrogen (7.2.4): uses Green's function; the Dirac distribution compensates for the source in the equation; the Heaviside step function makes sure the neutrons scattered at least once scattered beyond, not below, the original lethargy. 
\item Slowing down with resonance absorptions (which is low energy) (7.2.5): we can approximate resonance absorption of the following types
  \begin{enumerate}
    \item Black resonance trap: 
      \eqn{ 1 - p = \int_0^{\gamma} \int_{u-\epsilon}^0 \du' \frac{1}{\xi} \frac{e^{-(u-u')}}{1 - \alpha} = \frac{1 - e^{-\gamma} - \alpha \gamma}{\xi (1-\alpha)}  }
      when $\gamma$ is small, we can simplify the above to be $1 - p = \frac{\gamma}{\xi}$. 
    \item Narrow grey resonance trap:
      \eqn{ 1 - p  = \int_{\mathrm{trap}} \frac{\Sigma_a(u)}{\Sigma_t(u)} \frac{\du}{\xi} }
    \item A set of narrow grey resonance traps: we express each resonance in exponential form, and find the product of them:
      \begin{align}
        p_i &= 1 - \int_i \frac{\Sigma_a(u)}{\xi \Sigma_t (u)} \du = \exp \left[ -\int_i \frac{\Sigma_a(u)}{\xi \Sigma_t(u)} \du \right] \\
        p &= \prod_i p_i = \exp \left[ - \Sum_i \int_i \frac{\Sigma_a(u)}{\xi \Sigma_t(u)} \du \right]
      \end{align}
    \item General expression:
      Because the integrated function is zero outside of the resonance traps, we can simply write,
      \eqn{p =  \exp \left[ - \int \frac{\Sigma_a(u)}{\xi \Sigma_t(u)} \du \right] }
      This is the general expression for resonance escape probability. 
  \end{enumerate}
\item Slowing down with low, slowly varying absorption (which is high energy) (7.2.6). 

  \begin{enumerate}
  \item Greuling-Goertzel approximations for slow and gradually varying absorption: 
  \item Wigner approximation for resonance absorption: 
  \end{enumerate}
\end{enumerate}

%%%%%%%%%%%%%%%%%%%%%%%%%%%% Ch 9 %%%%%%%%%%%%%%%%%%%%%%%%%%%%%%%%
\topic{Summary: Reuss Ch 9 Thermalisation of Neutrons}
\begin{enumerate}
\item For monatomic gas, there is thermal agitation and no chemical bonds, we can approximaet with Maxwell distribution. 
\item For any thermaliser, the neutron spectrum at equilibrium and in the absence of absorption (or with low capture) would be a Maxwell spectrum. A Maxwell spectrum depends on $E, kT$. 
\item Microreversibility principle/detailed balance: at equilibrium and with no absorption, the number of transfers by diffusion from $\dE$ to $\dE'$ as transfers in the other direction from $\dE'$ to $\dE$. 
\item Scattering equation (9.1.4): 
\eqn{ \Sigma_s(E', E, \mu) = \Sigma_s (E') P(E' \to E) P(\mu) = C^{te} \sqrt{\frac{E}{E'}} \exp \left[\frac{E'-E}{2kT} \right] S(\alpha, \beta)  }
\item Thermalisation equation (9.1.5): 
  \eqn{ \int_0^{E_{\mathrm{cutoff}}} \Sigma_s (E') \Phi(E') \dE' P(E' \to E) + \overbrace{\int_{E_{\mathrm{cutoff}}}^{\infty} \Sigma_s(E') \Phi(E') \dE' P(E'\to E)}^{S_{\mbox{slow-down}}(E)} = \Sigma_t(E) \Phi(E) }
  and the slow-down equation, 
  \eqn{ \int_{-\infty}^{u} \Sigma_s (u') \overbrace{P(u' \to u)}^{\frac{\exp(-(u-u'))}{1 - \alpha}} \Phi(u') \du' + S(u) = \Sigma_t(u) \Phi(u) }
  The two are similar, except:
  \begin{enumerate}
  \item The slowing down equation only down scatters, whereas thermalisation equation can transfer energies in both directions. The upper boundary $E_{\mathrm{cutoff}}$ is the energy separating the thermalization domain from the slowing down domain. 
  \item The `source' in the thermalisation equation $S_{\mbox{slow-down}}(E)$ is not a true source; it is a density of arrival at energies below the cutoff energy due to scattering events occurring in the last part of the slowing down domain and transferring the neutron beyond the cutoff energy in the thermalisation domain. 
  \end{enumerate}
\item Thermal Spectrum(9.2): it is approximately reasonable well by Maxwell distribution, except,
  \begin{itemize}
    \item $0\sim 2kT$: real thermal spectrum is smaller due to absorption of neutrons;
    \item $2kT$ and above: Maxwell spectrum approaches zero quickly, whilst the real density falls slightly but remains significant, due to the `slowing down queue:' neutrons coming from high energies slow down and enter the thermal domain, compensating for the disappearances by absorption. 
  \end{itemize}
\item Comparing MOX and UOX spectrum: the two are close for the fast and epithermal domains, because they essentially have the same quantity of moderator, U238, and cladding. Whereas in the thermal domain, the number of neutrons in MOX is lower by a factor of 4 because of the high absorption by MOX fuel of thermal neutrons. 
\item Average cross section (9.2.3): 
  \eqn{ \bar{\sigma} = \frac{\int \sigma(E) \Phi(E) \dE}{\int \phi(E) \dE} }
  \uline{Example: calculate the average xs for a Maxwell spectrum and a 1/v xs.} 
  \eqn{ \bar{\sigma} = \frac{\sqrt{\pi}}{2} \sigma(v_0) = \frac{\sqrt{\pi}}{2} \sqrt{\frac{293}{T}} \sigma_{2200} }
  where $\frac{\sqrt{\pi}}{2}$ is the average of $x = v/v_0$ on a Maxwell spectrum, and also the average of $1/x$. 
\item Heterogeneous Configurations (9.2.4):
  \begin{align}
    V_f R_f \Phi_f P_{ff} + V_m (R_m \Phi_m + S_{\mathrm{sl-d}} ) P_{mf} &= V_f \Sigma_f \Phi_f \\
    V_fR_f \Phi_f P_{fm} + V_m(R_m \Phi_m + S_{\mathrm{sl-d}}) P_{mm} &= V_m \Sigma_m \Phi_m 
  \end{align}
\item Approximate thermal neutron speeds(9.3.1): assume absorption xs are 1/v and scattering xs are constant, we can approximate
  \eqn{ v = 2200 \m/\s \sqrt{\frac{T}{293.15}} } 
\item Thermal utilisation factor (9.3.2): the fraction of thermal neutrons absorbed in the fuel,
  \eqn{ f &= \frac{V_f \Sigma_{a,f} \Phi_f}{V_f \Sigma_{a,f} \Phi_f + V_m \Sigma_{a,m} \Phi_m + \cdots}, & \frac{1}{f} - 1 &= \underbrace{\frac{V_m}{V_f}}_{\mbox{moderation ratio}}
    \overbrace{\frac{\Phi_m}{\Phi_f}}^{\mbox{disadvantage factor}} \frac{\Sigma_{a,m}}{\Sigma_{a,f}}   } 
\item Reproduction factor $\eta$ (9.3.3):
  \eqn{ \eta = \frac{\nu \Sigma_{f,f}}{\Sigma_{a,f}} }
\end{enumerate}
%%%%%%%%%%%%%%%%%%%%%% end of Ch 9 %%%%%%%%%%%%%%%%%%%%%%%%%%%%%


\topic{Developing An Infinite-medium Monte Carlo Neutronics}
\subtopic{Isotopic Importance by Infinite Medium Reactor Materials}
\begin{table}
  \centering
  \begin{tabular}{|c|c|c|c|} \hline
    Reactor & Fuel & Coolant \& Moderator & Structural  \\ \hline \hline
    PWR & 35\% & 55\% & 10\% \\ \hline
    CANDU6 & 5\% & 85\% & 10\%  \\ \hline
    HTGR & 5\% & 90\% & 2\% \\ \hline
    SFR & 60\% & 30\% & 10\% \\ \hline
  \end{tabular}
  \caption{Volume Fraction/Isotopic Importance of Reactor Materials} \label{volume-fraction}
\end{table}
For a typical LWR, we can find common isotopes' number densities by using the volume fraction in Table~\ref{volume-fraction}: 
\begin{align}
N^H_c &= 0.55 \times 1 \g/\cm^3 \frac{N_{AV} \mathrm{molecule}/\mol}{18 \g/\mol} 2/\mathrm{molecule}  = 3.68 \times 10^{22} \\
N^O_c &= 0.55 \times 1 \g/\cm^3 \frac{N_{AV} \mathrm{molecule}/\mol}{18 \g/\mol} 1/\mathrm{molecule}  = 1.84 \times 10^{22} \\
N^O_f &= 0.35 \times 10 \g/\cm^3 \frac{N_{AV} \mathrm{molecule}/\mol}{270 \g/\mol} 2/\mathrm{molecule}  = 1.56 \times 10^{22} \\
N^{238}_f &= 0.95 \times 0.35 \times 10 \g/\cm^3 \frac{N_{AV} \mathrm{molecule}/\mol}{270 \g/\mol} 1/\mathrm{molecule}  = 1.48 \times 10^{22} \\
N^{235}_f &= 0.05 \times 0.35 \times 10 \g/\cm^3 \frac{N_{AV} \mathrm{molecule}/\mol}{267 \g/\mol} 1/\mathrm{molecule}  = 0.079 \times 10^{22} \\
N^{Zr} &= 0.10 \times 6.6 \g/\cm^3 \frac{N_{AV} \mathrm{molecule}/\mol}{90 \g/\mol} 1/\mathrm{molecule}  = 0.56 \times 10^{22} 
\end{align}
That is, in a typical LWR the relative number densities come out to be around: 
\eqn{\boxed{H = 1, O = 1, U238 = 0.4, U235 = 0.02, Zr = 0.15   }}
Notice hydrogen's and oxygen's number densities are approximately the same. 

\subtopic{Absorption and Scattering Cross Sections}
See Figure~\ref{scatter-xs}, Figure~\ref{capture-xs} for common cross sections. Notice absorption and scattering tend to line up in terms of energy. 

\uline{Practice: Why can we use sodium as coolant if its capture cross section is so high?}  

\textbf{Answer:} Sodium has a high capture xs in thermal range, and a small capture xs in fast range. But we can still build a thermal reactor with water as moderator and sodium as coolant, as long as sodium's volume is lower than that of the H. 


\clearpage
%%%%%%%%%%%% Elastic Scattering %%%%%%%%%%%%%%%%%%%%%%
\topic{Models for High Energy Elastic Scattering Physics}
Assume isotropic scattering in COM (target at rest in Lab system), then the minimum final energy is,
\eqn{ \frac{E_{\mathrm{min}}}{E_0} = \alpha = \left( \frac{A-1}{A+1} \right)^2 } 
We define \hi{mean log energy decrement},
\eqn{ \zeta = 1 - \frac{\alpha \ln(\alpha)}{1 - \alpha}  } 
whose common values are: $\zeta_{H} = 1, \zeta_D = 0.725, \zeta_C = 0.158, \zeta_U = 0.0084$. Then the number of collisions required to slow a particle from $E_2$ to $E_1$ is: 
\eqn{ n = \frac{\ln(E_2 / E_1)}{\zeta} }
For $E_2 = 2\times 10^6 \eV, E_1 = 0.1 \eV$, the number of collisions is: $n_H = 17, n_D = 23, n_C = 106, n_U = 2000$. 

\subtopic{Result: Flux vs. Energy}
\begin{figure}
  \centering
  \includegraphics[width=3in]{images/sl-d/flux-vs-energy-1.png}
  \includegraphics[width=3in]{images/sl-d/flux-vs-energy-2.png}
  \caption{Flux vs. Energy} \label{fve}
\end{figure}
There are two ways to understand flux here. 
\begin{itemize}
\item Particle based: we let all the neutrons to scatter till death; then for each energy interval, if we add up the number of neutrons with energies in that interval; we get the flux corresponding to each interval.  
\item Generation based: we let each neutron scatter once in each generation, save all the number of neutrons vs. energy data, and add up all the generations, this would effectively give us flux vs. energy as well[FIXME]. 
\end{itemize}
Figure~\ref{fve} presents the results from th generation-based flux. Notice with insufficient number of generations (left plot), the flux tails down at lower energies, meaning that some neutrons are not slowed down entirely yet. With sufficient number of generations (right plot), we get a \hi{1/E spectrum}. 

\subtopic{Result: Flux vs. Lethargy}
\begin{figure}
  \centering
  \includegraphics[width=3in]{images/sl-d/flux-vs-lethargy-1.png}
  \includegraphics[width=3in]{images/sl-d/flux-vs-lethargy-2.png}
  \caption{Flux vs. Lethargy} \label{fvl}
\end{figure}
A 1/E spectrum in energy space is a constant value in lethargy space as in Figure~\ref{fvl}. 

Caution about number of bins: When we are plotting, we are really plotting the expectation value of the numbers of neutrons; so if we ask `how many neutrons would there be at x energy level' the answer should be zero. So the bins have to be fine enough to see the details, but not too fine that the bins got no more neutrons. 

\clearpage
\topic{Models for Fission Neutron Emission Spectrum ($\chi$)}
Model: Maxwell spectrum:
\eqn{ \chi(E) \dE = \frac{2\pi}{(\pi T)^{3/2}} \sqrt{E} \exp\left( - \frac{E}{T} \right) \dE }
Simulation: constructing a cdf for Maxwell distribution (when constructing any cdf, it is a good practice to sample immediately and double check with the pdf); for each neutron, generate a random number $\xi \in [0,1]$, and use $\xi = cdf(E)$ to reverse-lookup for $E$, and this is the fission emission energy of that neutron. 

Characteristics of our spectrum so far: a spectrum is the shape in energy; the spectrum is independent on the number of incoming neutrons; for one specie, the spectrum is independent of its cross section; for two species, only the relative cross section matters. 
\begin{figure}
  \centering
  \includegraphics[width=3in]{images/sl-d/spec-1.uncrop.pdf}
  \includegraphics[width=3in]{images/sl-d/spec-2.uncrop.pdf}
  \includegraphics[width=3in]{images/sl-d/spec-3.uncrop.pdf}
  \caption{Spectrum, Added in Fission Neutron Emission Spectrum and Hydrogen Scattering XS} \label{spe1}
\end{figure}

Fission source peak: so far we reach a flat spectrum. To obtain the peak around 2 MeV (fission source peak), we have to divide the flux count by the hydrogen cross section:
\eqn{\phi(E) \propto \Sum_N \Sum_g N_g^n \tau_g^n = \Sum_n \Sum_g N_g^n \frac{1}{\Sigma(E_g^n)}   }
Recall the hydrogen scattering cross section decreases from 0 to 1 eV, constant between 1 eV to 10 keV, and decrease afterwards. In the fast energies, the decreased hydrogen absorption cross section and the neutron fission emission spectrum (peaks at 1.7 MeV, average at 2 MeV). 


\clearpage
\topic{Models for Equilibrium Thermal Scattering Physics}
\subtopic{Elastic Scattering for Monatomic (Maxwellian) Free Gas}
Maxwellian free gas has elastic scattering cross section dependent on A, energy, and temperature:
\begin{align}
  \sigma(E) f_s (E \to E') = &\frac{\sigma_{s0} \eta^2}{2E} \left[ \erf{\eta \sqrt{\frac{E'}{kT}} - \rho \sqrt{\frac{E}{kT}} } \mp \erf{\eta \sqrt{\frac{E'}{kT}} + \rho \sqrt{\frac{E}{kT}} } \right. \\
    & \left. + e^{\frac{E - E'}{kT}} \left( \erf{ \eta \sqrt{\frac{E}{kT}} - \rho \sqrt{\frac{E'}{kT}} }  \pm \erf{\eta \sqrt{\frac{E}{kT}} + \rho \sqrt{\frac{E'}{kT}} } \right) \right] 
\end{align}
where the upper signs are for $E' > E$< and the lower signs for $E' < E$, and the $\eta, \rho$ terms are:
\eqn{ \eta &= \frac{A+1}{2 \sqrt{A}} & \rho &= \frac{A-1}{2 \sqrt{A}} }
For proton gas (A=1), we can simplify the energy transfer function to be,
\begin{align}
\sigma_s(E) f_s(E\to E') = \frac{\sigma_{s0}}{E} \left\{ 
\begin{array}{cc}
e^{\frac{E-E'}{kT}} \erf{\sqrt{\frac{E}{kT}}} & E' > E \\
\erf{\sqrt{\frac{E'}{kT}}} & E' < E 
\end{array}
\right.
\end{align}
Two samples are plotted in Figure~\ref{ts-C-H}. Notice that a) E = 1kT to 25kT provides a good cover of range, and even at 1200K, 1kT = 0.1 eV, and 25kT = 2.5eV, so 4eV is a typical upper scattering cutoff; b) the $\frac{E'}{E}$ axis shows that there is a good probability that the neutron would gain energy from the elastic scattering(the lower the initial neutron energy, the closer it is to the target nuclei thermal agitation, the more probable is upper scattering). In fact this is the difference between this thermal energy model and the previous high-energy model.  
\begin{figure}
  \centering
  \includegraphics[width=3in]{images/sl-d/ts_H.uncrop.pdf}
  \includegraphics[width=3in]{images/sl-d/ts_C.uncrop.pdf}
  \caption{Thermal Scattering Probability vs. Energy for Hydrogen and Graphite} \label{ts-C-H}
\end{figure}

Adding the thermal scattering into our slowing down and thermalization in hydrogen model, we observe a peak in the thermal range as well. To make the height more realistic, we have to add in the hydrogen absorption cross section (a 1/v distribution, only affect the thermal range). The absorbing strength is picked such that the absorption-to-scattering ratio is something reasonable, like 0.2. 
\begin{figure}
  \centering
  \includegraphics[width=3in]{images/sl-d/spec-4.uncrop.pdf}
  \includegraphics[width=3in]{images/sl-d/spec-5.uncrop.pdf}
  \caption{Spectrum, Added in Thermal Scattering and 1/v Hydrogen Absorption XS} \label{spe1}
\end{figure}

Notice that in the above thermal scattering model, we made two assumptions:
\begin{itemize}
\item The target is at rest;
\item Free target. 
\end{itemize}
In the next two subsections, we are going to cover when do the above conditions fail. 

\subtopic{Adding Target in Motion: Important for U238 Resonances at Hot Fuel Conditions}
Thermal agitation does not matter that much for non-resonance conditions; that is, when the cross section changes reasonably slow, a small $\Delta v$, hence a small $\Delta E$ does not change the cross section much. This is not true anymore for U238 resonances, because when cross section changes very rapidly (especially true in the scattering cross section than the absorption cross section), target motion/thermal agitation would affect up scattering quite significantly. As shown in Figure~\ref{timr}, it could be a 10\% increase in LWR U238 Doppler.  
\begin{figure}
  \centering
  \includegraphics[width=4in]{images/sl-d/target-in-motion-resonance.png}
  \caption{Target Motion Is Important for U239 Resonances} \label{timr}
\end{figure}
We will cover resonance models later, for now we just need to know that the weird dips in scattering cross section is a result of the interacting wave between the potential scattering and the compound nuclear scattering. 

Caution about cutoff energy: the resonance scattering cross section changes significantly around the resonance energies, and if our cutoff energy is placed right before the peak, the results would be wrong. 


\subtopic{Adding Chemical Bound: Bound Elastic Scattering of H in Water Molecule vs Temperature}
So far we've only talked about free gas; in the case of tightly bound atoms at very low energy, the vibrations within a molecule due to the chemical bond come into play:
\eqn{\sigma_{\mathrm{bound}} = \left( 1 + \frac{1}{A_{\mathrm{atom}}} \right) \sigma_{\mathrm{free}}  }
For bound water molecule for instance, 
\eqn{ \sigma_{\mathrm{water}}^H = (1+1)^2 \sigma_H = 4 \sigma_H }
The dependency on $A$ suggests that \textit{the bound elastic scattering matters for light nuclei, and not so much for heavy nuclei}. In the case of hydrogen, the free gas model would still provide the right shape, but the probability can be off by a factor of 10. In the case of graphite, the free gas model does not provide the right shape. The lower energy you want to go, the more careful you need to be. 

Elastic scattering for bound molecules can be characterized by,
\eqn{ \frac{\derivative^2 \sigma}{\dOmega \dE'} (E\to E', \Omega\to \Omega')  = \frac{\sigma_b}{4 \pi kT} \sqrt{ \frac{E'}{E}} e^{-\frac{\beta}{2}} S(\alpha, \beta)   }
where $\sigma_b$ is the bound scattering cross section for the material, $kT$ is in eV, $S(\alpha, \beta)$ is the symmetric form of the thermal scattering law,
\eqn{ S(\alpha, \beta) = \frac{\exp{-\frac{\alpha^2 + \beta^2}{4 \alpha}} }{\sqrt{4\pi \alpha}} }
where $\alpha, \beta$ depends on two terms:
\begin{itemize}
\item The momentum transfer $\kappa$,
\eqn{ \alpha = \frac{E' + E - 2 \sqrt{E' E} \cos \theta}{A kT} = \frac{\hbar^2 \kappa^2}{2 M kT} } 
where $A$ is the ratio of the mass of the scattering atom to the neutron mass.
\item The energy transfer $\epsilon$,
\eqn{ \beta = \frac{E' - E}{kT} = \frac{\epsilon}{kT} }
\end{itemize}


\subtopic{Recap}
A couple of important points:
\begin{itemize}
\item Thermal scattering distributions: generate cdf for energies, like from 0.1kT to 8kT.  
\item 4 eV is the typicall up-scattering cutoff energy. For an incident neutron energy higher than 4 eV, use asymptotic elastic down scattering. 
\end{itemize}

\end{document}

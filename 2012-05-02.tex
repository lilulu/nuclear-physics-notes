\documentclass{school-22.211-notes}
\date{May  2, 2012}

\begin{document}
\maketitle

\lecture{Nodal Diffusion Methods}
\topic{3D Core Analysis Overview}
Reference: \href{www.oecd-nea.org/dbprog/documents/MC03Smith.pdf}{`Reactor Core Methods' by Smith, MC 2003}. 
There are a couple levels of complexity for performing 3D core analysis,
\begin{itemize}
\item One reactor at one state point: couple core neutronics/fuel heat conduction/coolant hydraulics/structural mechanics for 60,000 fuel pins and 100 axial levels to evaluate thermal margins;
\item Repeat for 50 depletion states and track 300 isotopes;
\item Repeat for 10,000s limiting conditional simulations and 100s of transient accident simulations for safety analysis;
\item Repeat for 1000s of startup and maneuver simulations;
\item Repeat for millions of cycle depletion simulation for designing loading design and optimization;
\item For operator training simulator in real time, 24/7 operation at 5-10 Hz. 
\end{itemize}
The challenges for 3D core calculation include,
\begin{itemize}
\item Predict pin power, axial shapes of pin power: as in Figure~\ref{3Dcore}, the spacers depress the flux. Though even without the spacers, the axial shape would still be more towards the bottom compare with a cosine shape due to the temperature coefficients; 
\item Predict core reactivity with core burnup: as in Figure~\ref{3Dcore}, burnable poisons deplete as burnup increases;
\item Predict in-core detector response;
\item Predict control rod worths and temperature coefficients. 
\end{itemize}
\begin{figure}[ht]
  \centering
  \includegraphics[width=3in]{images/methd/3Dcore-spacer.png}
  \includegraphics[width=3in]{images/methd/3Dcore-poison.png}
  \caption{3D Core Calculation Samples} \label{3Dcore}
\end{figure}

Approaches to 3D core problems:
\begin{enumerate}
\item S$_N$ Method: computational time problem. We are looking at 100 mesh per pin, 60,000 pins per core, 100 axial mesh, 100 energy groups, 1000 angles, that's about $6 \times 10^{13}$ unknowns. So even with perfect scaling, we are looking at 100,000 core hours for a full core, not to mention feedback, cross section evaluations, boron and control rod searches to critical, etc. Hence S$_N$ has never been done for a single LWR core. 

\item Pin-by-pin diffusion with homogenized pin cells: accuracy problem. There are only $6 \times 10^8$ unknowns, because we only have 1 mesh per pin instead of 100, and 1 angle instead of 1000. It brings the computing time down to 1 hour, but pin homogenization causes a loss of accuracy around strong absorbers. This is now a tractable problem, but not used for production.

\item Nodal method: this is what production codes typically use. There is a huge push towards nodal method, because nodal methods take a mesh size of 20cm to be accurate, whereas Finite Difference method requires a mesh size of about 1 cm. 
\end{enumerate}

\clearpage
\topic{Derivation of Nodal Methods}
Reference: Sutton and Aviles' `Diffusion Theory Methods for Spatial Kinetics Calculations' (1996). 
\begin{enumerate}
\item We start from 3D steady-state multigroup neutron diffusion equation, 
  \eqn{\divergence \vecJ_g(\vecr, E) + \Sigma_{rg} \psi_g (\vecr, E) = \frac{1}{\keff} \chi_g \Sum_{g' = 1}^G \nu \Sigma_{fg'} \psi_{g'}(\vecr, E) + \Sum_{g'=1}^G \Sigma_{sg'g} \psi_{g'}(\vecr, E) }
  Apply Fick's Law of diffusion for current out of flux, 
  \eqn{ \vecJ_g(\vecr, E) = - D_g(\vecr, E) \gradient \psi_g(\vecr, E) }

\item Assumptions: properties are constant in homogenized nodes. Then we find the volume-averaged terms for each node; that is, we integrate over the node volume and then divide by volume,

\item Volume average of diffusion equation for a node: we integrate every term over the node volume then divide by volume. The volume averaged flux becomes, 
  \eqn{ \bar{\psi} = \frac{1}{h_x h_y h_z} \int_0^{h_x} \int_0^{h_y} \int_0^{h_z} \psi(x,y,z) \dx \dy \dz }
  The leakage term becomes (after integrating the divergence term using Gauss Theorem), 
  \eqn{ \int_0^{h_x} \int_0^{h_y} \int_0^{h_z} \divergence \vecJ_g \dx \dy \dz &= \int_0^{h_y} \int_0^{h_z} (J_x(h_x,y,z) - J_x(0,y,z)) \dy \dz  \\
&+ \int_0^{h_x} \int_0^{h_z} (J_y (x,h_y, z) - J_y(x,0,z)) \dx \dz \\
&+ \int_0^{h_x} \int_0^{h_y} (J_z(x,y,h_z) - J_z(x,y,0)) \dx \dy }
  If we define surface-averaged currents, 
  \eqn{ \bar{J}_{gxl} &= \frac{1}{h_y h_z} \int_0^{h_y} \int_0^{h_z} J_x(0,y,z) \dy \dz, & \bar{J}_{gxr} &= \frac{1}{h_y h_z} \int_0^{h_y} \int_0^{h_z} J_x(h_x,y,z) \dy \dz }
  Then the leakage term becomes, 
  \eqn{ \Sum_{u=x,y,z} \frac{\bar{J}_{gur} - \bar{J}_{gul}}{h_u} }
  Together, the nodal balance equation with average terms is, 
  \eqn{ \Sum_{u=x,y,z} \frac{\bar{J}_{gur} - \bar{J}_{gul}}{h_u} + \Sigma_{rg} \bar{\psi}_g &= \frac{1}{\keff} \chi_g \Sum_{g'=1}^G \nu \Sigma_{fg'} \bar{\psi}_{g'} + \Sum_{g'=1}^G \Sigma_{sg'g} \bar{\psi}_{g'} \label{nodal-balance} }
  Notice, 
  \begin{itemize}
    \item The six surface-averaged currents are required to find the node-averaged flux which would determine nodal power;
    \item Surface-averaged currents are average of flux derivative on a surface, which equals derivative of average flux at the surface; 
    \item It is more efficient to work with diffusion equation for average flux rather than solving the diffusion equation for the point-wise fluxes in 3D. 
  \end{itemize}

\item Transverse Integration: we need a set of equations for the surface average currents instead of solving 3D finite different equations directly. To do so, we pick a direction of interest (eg., x direction), and perform integration within node over 2D plane normal to that direction (eg., the grey surface in Figure~\ref{nodal-current}), then devide by the planar area. The LHS of Eq.~\ref{nodal-balance} becomes, 
  \eqn{\Sum_{u=x,y,z} \frac{\bar{J}_{gur} - \bar{J}_{gul}}{h_u} + \Sigma_{rg} \bar{\psi}_g = \frac{1}{h_y h_z} \int_0^{h_z} \int_0^{h_y} (\divergence \vecJ_g(\vecr, E) + \Sigma_{rg} \psi_g (\vecr, E) ) \dy \dz }
\begin{figure}[ht]
  \centering
  \includegraphics[width=0.28\textwidth]{images/methd/nodal-current.png}
  \caption{Integrate Within Node Over 2D Plane Normal to x-direction} \label{nodal-current}
\end{figure}
  \begin{enumerate}
  \item Define dimensionless independent variables,
    \eqn{ \xi_x &= \frac{x}{h_x}, & \xi_y &= \frac{y}{h_y}, & \xi_z &= \frac{z}{h_z} }
    and we can transform the integration and the derivative operator as well, 
    \eqn{ \frac{\du}{h_u} &= \derivative \xi_u, & \ddu &= \frac{1}{h_u}\frac{\derivative}{\dxi_u}, & u&= x,y,z}

  \item Simplify the averaging with dimensionless variables,
    \eqn{ J_{xgl} &= \int_0^1 \int_0^1 J_x(0,\xi_y, \xi_z) \dxi_y \dxi_z, &J_{gxr} &= \int_0^1 \int_0^1 J_x(1,\xi_y, \xi_z) \dxi_y \dxi_z }
    \eqn{ \bar{\psi} &= \int_0^1 \int_0^1 \int_0^1 \psi(\xi_x, \xi_y, \xi_z) \dxi_x \dxi_y \dxi_z }

  \item The normalized 3D diffusion equation becomes, 
    \eqn{ \left( - \Sum_{u=x,y,z} \frac{D_g}{h_u^2} \frac{\partial^2}{\partial \xi_u^2} + \Sigma_{rg} \right) \psi_g(\xi_x, \xi_y, \xi_z) &=  \frac{1}{\keff} \chi_g \Sum_{g'=1}^G \nu \Sigma_{fg'}\psi_g(\xi_x, \xi_y, \xi_z) + \Sum_{g'=1}^G \Sigma_{sg'g} \psi_g(\xi_x, \xi_y, \xi_z) }

  \item We look at the leakage term after the transverse integration, 
    \begin{align}
      &\int_0^1 \int_0^1 \divergence \vecJ \dxi_y \dxi_z = \int_0^1 \int_0^1 \left( \frac{1}{h_x} \frac{\partial J_x}{\partial \xi_x} + \frac{1}{h_y} \frac{\partial J_y}{\dxi_y} + \frac{1}{h_z} \frac{\partial J_z}{\dxi_z} \right) \dxi_y \dxi_z \\
      &= \int_0^1 \int_0^1 \left( -\frac{D_g}{h_x^2} \frac{\partial^2 \psi}{\partial \xi_x^2} + \frac{1}{h_y} \frac{\partial J_y}{\dxi_y} + \frac{1}{h_z} \frac{\partial J_z}{\dxi_z} \right) \dxi_y \dxi_z \\
      &= - \frac{D_g}{h_x^2} \frac{\partial^2 }{\partial \xi_x^2} \int_0^1 \int_0^1 \psi(\xi_x, \xi_y, \xi_z) \dxi_y \dxi_z \\
      &+ \frac{1}{h_y} \int_0^1 (J_y(\xi_x, 1, \xi_z) - J_y(\xi_x, 0, \xi_z) )\dxi_z + \frac{1}{h_z} \int_0^1 (J_z(\xi_x, \xi_y, 1) - J_z(\xi_x, \xi_y, 0) ) \dxi_y \\
      &= - \frac{D_g}{h_x^2} \frac{\partial \bar{\psi}_x (\xi_x)}{\partial \xi_x^2} + \frac{\bar{J}_{yr}(\xi_x) - \bar{J}_{yl} (\xi_x)}{h_y} + \frac{\bar{J}_{zr} (\xi_x) - \bar{J}_{zl} (\xi_x)}{h_z} 
    \end{align}
    where we defined the \textbf{plane-averaged 1D flux},
    \eqn{\bar{\psi}_x = \int_0^1 \int_0^1 \psi(\xi_x, \xi_y, \xi_z) \dxi_y \dxi_z }
    and \textbf{line-averaged surface current} at an arbitrary position $\xi_x$, 
    \eqn{ \bar{J}_{yr} (\xi_x) &= \int_0^1 J_y(\xi_x, 1, \xi_z) \dxi_z, &\bar{J}_{yl}(\xi_x) &= \int_0^1 J_y(\xi_x, 0, \xi_z) \dxi_z }
    \eqn{ \bar{J}_{zr} (\xi_x) &= \int_0^1 J_z(\xi_x, \xi_y, 1) \dxi_y, &\bar{J}_{zl}(\xi_x) &= \int_0^1 J_z(\xi_x, \xi_y,  0) \dxi_y }

  \item Collect all the terms, we get the transverse integration of 3D diffusion equation, 
    \eqn{ \Sum_{u=y,z} \frac{\bar{J}_{gur}(\xi_x) - \bar{J}_{gul}(\xi_x)}{h_u} - \frac{D}{h_x^2} \frac{\derivative^2}{\dxi_x^2} \bar{\psi}_{gx} (\xi_x) + \Sigma_{rg} \bar{\psi}_{gx} (\xi_x) &=  \frac{1}{\keff} \chi_g \Sum_{g'=1}^G \nu \Sigma_{fg'}\psi_g(\xi_x) + \Sum_{g'=1}^G \Sigma_{sg'g} \psi_g(\xi_x)  }
    If we define a \textbf{transverse leakage term} and move it to the RHS, 
    \eqn{ L_{gu}(\xi_x) &= \frac{1}{h_u} \left( \bar{J}_{gur}(\xi_x) - \bar{J}_{gul}(\xi_x) \right), & u &= y,z }
    and also define the \textbf{diffusion equivalent group constant}, 
    \eqn{ \Sigma_{Dg}^x  = \frac{D}{h_x^2} }
    We get the transverse integrated 1D diffusion equation,
    \eqn{ - \Sigma_{Dg}^x \frac{\derivative^2}{\dxi_x^2} \bar{\psi}_{gx} (\xi_x) + \Sigma_{rg} \bar{\psi}_{gx} (\xi_x) = \frac{1}{\keff} \chi_g \Sum_{g'=1}^G \nu \Sigma_{fg'} \bar{\psi}_{g'x} (\xi_x) + \Sum_{g'=1}^G \Sigma_{sg'g} \bar{\psi}_{g'x} (\xi_x) - L_{gy} (\xi_x) - L_{gz} (\xi_x)  } 

  \item Repeat for the other two directions, we get a set of 3 directional 1D diffusion equations.
    \begin{align}
      \boxed{ - \Sigma_{Dg}^x \frac{\derivative^2}{\dxi_x^2} \bar{\psi}_{gx} (\xi_x) + \Sigma_{rg} \bar{\psi}_{gx} (\xi_x) = \frac{1}{\keff} \chi_g \Sum_{g'=1}^G \nu \Sigma_{fg'} \bar{\psi}_{g'x} (\xi_x) + \Sum_{g'=1}^G \Sigma_{sg'g} \bar{\psi}_{g'x} (\xi_x) - L_{gy} (\xi_x) - L_{gz} (\xi_x) } \notag \\
   \boxed{ - \Sigma_{Dg}^y \frac{\derivative^2}{\dxi_y^2} \bar{\psi}_{gy} (\xi_y) + \Sigma_{rg} \bar{\psi}_{gy} (\xi_y) = \frac{1}{\keff} \chi_g \Sum_{g'=1}^G \nu \Sigma_{fg'} \bar{\psi}_{g'y} (\xi_y) + \Sum_{g'=1}^G \Sigma_{sg'g} \bar{\psi}_{g'y} (\xi_y) - L_{gz} (\xi_y) - L_{gx} (\xi_y) }  \notag \\
     \boxed{ - \Sigma_{Dg}^z \frac{\derivative^2}{\dxi_z^2} \bar{\psi}_{gz} (\xi_z) + \Sigma_{rg} \bar{\psi}_{gz} (\xi_z) = \frac{1}{\keff} \chi_g \Sum_{g'=1}^G \nu \Sigma_{fg'} \bar{\psi}_{g'z} (\xi_z) + \Sum_{g'=1}^G \Sigma_{sg'g} \bar{\psi}_{g'z} (\xi_z) - L_{gx} (\xi_z) - L_{gy} (\xi_z) } \notag
    \end{align}

  \item Interpretations: we turn a 3D partial differential equation into three 1D ordinary differential equation that are coupled through average transverse leakage term. It is exact if the transverse leakage shape is known. 
  \end{enumerate}

\item Approximation on transverse leakage: from observation, flux is insensitive to the transverse leakage shape, so we are going to perform a quandrature polynomial with 2nd order polynomials of the leakage term, and iteratively update them. Quadratic approximation in each node, 
  \eqn{ L(\xi) = \bar{L} + l_1 P_1(\xi) + l_2 P_2(\xi) }
  We apply average TL conservation scheme to determine $l_1, l_2$. We use three node average transver leakages (the values of each node and its two adjacent nodes), and impose constraint of conserving the averages of two adjacent nodes. About how we handle the rest of this method, we present three methods in the following section. 
\end{enumerate}


%%%%%%%%%%%%%%%%%%%%%%%%%%
\clearpage
\topic{Three Nodal Methods}
\begin{enumerate}
\item Nodal Expansion Method (NEM) developed by Finnemann (KWU 1975). 
  \begin{enumerate}
  \item We approximate 1D flux by 4th order polynomial, 
    \eqn{ \bar{\psi}(\xi) = \Sum_{i=0}^4 a_i P_i (\xi) }
    where the basis functions are, 
    \begin{align}
      P_0(\xi) &= 1 \\
      P_1(\xi) &= 2 \xi - 1\\
      P_2(\xi) &= 6 \xi (1-\xi) - 1 \\
      P_3(\xi) &= 6 \xi (1-\xi)(2\xi-1) \\
      P_4(\xi) &= 6 \xi (1-\xi)(5\xi^2 -5\xi +1)
    \end{align}
    Notice these basis functions are not orthogonal, and integration from 0 to 1 would result in zero. Keep in mind that our transverse leakage is 2nd order (because the result is not very sensitive to the expansion of the transverse leakage). 

  \item BCs: In a two nodes case, we have 16 unknowns from: 4 flux moments $\times$ 2 nodes $\times$ 2 energy groups. But we only have 8 knowns, 
    \begin{itemize}
      \item 4 node average fluxes: 2 groups $\times$ 2 nodes;
      \item 2 interface continuity conditions: from 2 energy groups;
      \item 2 current continuity conditions: from 2 energy groups. 
    \end{itemize}
    Hence we need 8 additional constraints because polynomials can not satisfy the differential equations exactly. 

  \item To find the next 8 constrains, we use \textbf{weighted residual method} for each group in the 1D diffusion equation, 
    \eqn{ \int_0^1 w(\xi) \left( -\Sigma_D \frac{\derivative}{\dxi^2} \psi(\xi) + \Sigma_r \psi(\xi) \right) \dxi = \int_0^1 w(\xi) \left( \frac{1}{\keff} \chi \psi(\xi) + S(\xi) - L(\xi) \right) \dxi }
    The closure relationship basically forces the $P_1, P_2$ integration to go to zero. Forcing the 1st spatial moment of diffusion equation to go to zero gives us, 
    \eqn{ \int_0^1 P_1(\xi) \left( - \Sigma_D \frac{\derivative}{\dxi^2} \psi(\xi) + \Sigma_r \psi(\xi) - Q(\xi) \right) \dxi &= 0, & a_3 &= \frac{5 q_1 + 3 q_3 - 5 a_1 \Sigma_r}{3(60 \Sigma_D + \Sigma_r)} }
    Forcing the 2nd spatial moment of diffusion equation to go to zero gives us, 
    \eqn{ \int_0^1 P_2(\xi) \left( - \Sigma_D \frac{\derivative}{\dxi^2} \psi(\xi) + \Sigma_r \psi(\xi) - Q(\xi) \right) \dxi &= 0, & a_4 &= \frac{-7 q_2 + 3 q_4 + 7 a_2 \Sigma_r}{420(\Sigma_D + 3\Sigma_r)} }    
    Notice we don't know $a_1, a_2$. But we get two pieces of information from the spatial moments. That gives us in total 2 spatial moments $\times$ 2 nodes $\times$ 2 groups $= 8$ constraints. Together that's 16 constraints for 16 unknowns. 

  \item The choice of the weighted residual equations is flexible -- the results are not very sensitive to the weights. For instance, we can choose the same expansion coefficients for $P_i$. 

  \item Two-node 2-group NEM is different from finite-difference methods in terms of: 
    \begin{itemize}
      \item Flux shapes in each group affects coupling equations for other group;
      \item The coupling equations depend on transverse leakages;
      \item Final form of the equations can be made similar to 3D finite difference. 
    \end{itemize}
  \end{enumerate}


\clearpage
\item 2-Group Analytic Nodal Method (ANM) developed at MIT by Henry (1978). 
  \begin{enumerate}
  \item We start from the 1D 2-Group transverse-integrated diffusion equations, and move all terms except the transverse leakage on the LHS, 
    \begin{align}
      -D_1 \frac{\derivative^2 \psi_1(x)}{\dx^2} + \Sigma_{r1} \psi_1(x) - \frac{1}{\keff} (\nu \Sigma_{f1} \psi_1 (x) + \nu \Sigma_{f2} \psi_2 (x) ) &= - L_1(x) \\
      -D_2 \frac{\derivative^2 \psi_2(x)}{\dx^2} + \Sigma_{r2} \psi_2(x) - \Sigma_{12} \psi_1 (x) &= - L_2 (x) 
    \end{align}

  \item The analytic solution is in the form of, 
    \eqn{ \psi_g(x) = \psi_g^H (x) + \psi_g^P (x)}
    where the trial homogeneous solution is, 
    \eqn{ \psi_g^H (x) = \hat{\psi}_g^H \exp(i Bx) }
   
\item We find the homogeneous solution first. The characteristic equation is, 
    \eqn{ \left[ \begin{array}{cc} 
          D_1 B^2 + \Sigma_{r1} - \frac{1}{\keff} \nu \Sigma_{f1} & -\frac{1}{\keff} \nu \Sigma_{f2} \\ 
          -\Sigma_{12} & D_2 B^2 + \Sigma_{r2} 
          \end{array} \right] 
      \left[ \begin{array}{c} 
          \hat{\psi}_1^H \\ \hat{\psi}_2^H 
        \end{array} \right] = 
      \left[ \begin{array}{c} 0 \\ 0 \end{array} \right] }
    For nontrivial solution, we set the determinant to zero, and re-write the expression in terms of $B^2$: 
    \eqn{ (B^2)^2 + \overbrace{\left( \frac{\Sigma_{r1}}{D_1} + \frac{\Sigma_{r2}}{D_2} - \frac{\frac{\nu \Sigma_{f1}}{\keff}}{D_1} \right)}^{2b} B^2 + \overbrace{ \left( 1 - \frac{\kinf}{\keff} \right) \frac{\Sigma_{r1}}{D_1} \frac{\Sigma_{r2}}{D_2} }^{c} = 0 }
    The roots of characteristic equations are the \textbf{eigen-buckling}; more specifically, one of them is the \hi{fundamental mode}, and the other is the \hi{harmonic mode}\footnote{recall quadratic roots for $ax^2 + bx + c = 0$ are $x = \frac{-b \pm \sqrt{b^2 - 4ac}}{2a}$}, 
    \begin{align}
      &\mbox{Fundamental Mode} &B_1^2&= b\left(-1 + \sqrt{1 - \frac{c}{b^2}} \right) = \left\{ \begin{array}{cc} > 0 & \kinf > \keff \\ < 0 & \kinf < \keff \end{array} \right. \\
      &\mbox{Harmonic Mode} &B_2^2&=b\left(-1 - \sqrt{1 - \frac{c}{b^2}} \right) <0
    \end{align}
    The homogeneous solution for group $g$ is: 
     \begin{align}
      &\mbox{Fundamental Mode} &\psi_{g1}^H(x)&= \left\{ \begin{array}{cc} a_{g1} \sin(B_1 x) + a_{g2} \cos(B_1x) & \kinf > \keff \\ a_{g1} \sinh(B_1 x) + a_{g2} \cosh(B_1 x) & \kinf < \keff \end{array} \right. \\
      &\mbox{Second-Harmonic Mode} &\psi_{g2}^H&=a_{g3} \sinh(B_2 x) + a_{g4} \cosh(B_2 x) 
    \end{align}   
     The combined homogeneous solution shows that group 1 and group 2 equations are linearly dependent, 
     \eqn{ \left[ \begin{array}{c} \psi_1^H(x) \\ \psi_2^H (x) \end{array} \right] = \left[ \begin{array}{c} a_{11} \sin(B_1 x) + a_{12} \cos (B_1 x) \\ a_{21} \sinh(B_2 x) + a_{22} \cosh(B_2 x) \end{array} \right] =  \left[ \begin{array}{cc} r_1 & r_2 \\ 1 & 1 \end{array} \right] \left[ \begin{array}{c} a_{21} \sin(B_1 x) + a_{22} \cos (B_1 x) \\ a_{23} \sinh(B_2 x) + a_{24} \cosh(B_2 x) \end{array} \right] }
     where the fast-to-thermal flux ratio is defined as,
     \eqn{ r_m &= \frac{a_{11}}{a_{21}} = \frac{a_{12}}{a_{22}} = \frac{D_2 B_m^2 + \Sigma_{r2}}{\Sigma_{12}} }

\item Next we find the particular solution, which is determined solely by quadratic transverse leakage, 
  \eqn{ \psi_g^P (\xi) = c_{0g} + c_{1g} P_1(\xi) + c_{2g} P_2(\xi) }
  where  
\eqn{ \left[ \begin{array}{c} c_{1p} \\ c_{2p} \end{array} \right] &= A^{-1} \left[ \begin{array}{c} -b_{1p} \\ -b_{2p} \end{array} \right], &p &=1,2.
 &\left[ \begin{array}{c} c_{10} \\ c_{20} \end{array} \right] &= A^{-1} \left[ \begin{array}{c} -b_{10} + 6 D_1 c_{12} / \Delta x^2 \\ -b_{20} + 6 D_2 c_{22} / \Delta x^2 \end{array} \right]. }


\item Hence the general solution in a node is, 
     \eqn{ \left[ \begin{array}{c} \psi_1^H(x) \\ \psi_2^H (x) \end{array} \right] = \left[ \begin{array}{cc} r_1 & r_2 \\ 1 & 1 \end{array} \right] \left[ \begin{array}{c} a_{21} \sin(B_1 x) + a_{22} \cos (B_1 x) \\ a_{23} \sinh(B_2 x) + a_{24} \cosh(B_2 x) \end{array} \right] + \left[ \begin{array}{c} \psi_1^P(x) \\ \psi_2^P (x) \end{array} \right] }
     That is, we have 4 coefficients to determine for a 2 group problem. 
     
   \item BCs: quadratic transverse leakage for two nodes, node-average fluxes for two nodes. We have 8 unknown coefficients (4 per node times 2 nodes), and we have 9 constraints coming from: 
     \begin{itemize}
       \item 4 node-averaged fluxes: 2 groups times 2 nodes;
       \item 2 flux interface continuity from 2 energy groups;
       \item 2 current interface continuity from 2 energy groups.
     \end{itemize}
     Hence there is no need for weighted residual equations because we have the exact solution to the 1D diffusion equation. Then all groups are solved simultaneously.

   \item Pros and Cons: no need for weighted residual equations because we have exact solution to the 1D diffusion equation. The drawback is, all groups are solved simultaneously, so the generalization is very hard. But with the help of modal expansion, now we can do ANM for any number of energy groups. 
  \end{enumerate}

\clearpage
\item Semi-Analytic Nodal Method (SANM) developed at Studsvik (1985). 
  \begin{enumerate}
  \item SANM uses NEM equations for the fast group, and performs transverse integrated 1D neutron diffusion equation for the thermal group, 
    \eqn{ -D_2 \frac{\derivative^2}{\dx^2} \psi_2(x) + \Sigma_{r2} \psi_2(x) &= \Sigma_{s12} \psi_1(x) - L_{y2} (x) = Q_2 (x) }

  \item We approximate the source $Q$ with 4th order Legendre Polynomial, 
    \eqn{ Q_g(x) = \Sum_{i=0}^4 q_i P_i\left(\frac{2x}{h}\right) }
    
  \item The analytic solution of thermal group diffusion equation is, 
    \eqn{ \psi_g(x) &= A \sinh(\kappa_g x) + B \cosh(\kappa_g x) + \Sum_{i=0}^4 c_i P_i \left(\frac{2x}{h}\right), &\kappa_g&=\sqrt{\frac{\Sigma_g}{D_g}} }
    We end up with \textit{exponential homogeneous solution and polynomial particular solution}. 

\item We differentiate to get expression for net current at the interface: 
    \eqn{ J_g(x) &= -D_g \ddx \psi_g(x) }
    We use the continuity of net currents/flux to get analytic expression for coupling. 
    \begin{figure}[ht]
      \centering
      \includegraphics[width=5in]{images/methd/SANM-accuracy.png}
      \caption{Accuracy of SANM in Two-Group Application}
    \end{figure}
  \end{enumerate}
\end{enumerate}


\clearpage
\topic{Iterative Solution Technique}
In this section we show how the three nodal methods come down to be pretty much the same. The classic solution sequance involves,
\begin{enumerate}
\item Guess $\keff$, approximate transverse leakages as zero.
\item Evaluate all coupling terms for 1-node equation,
  \eqn{ \left[ \begin{array}{c} J_1 \\ J_2 \end{array} \right] = \left[ \begin{array}{cc} a_{11} & a_{12} \\ a_{21} & a_{22} \end{array} \right] \left[ \begin{array}{c} \bar{\psi}_1^+ \\ \bar{\psi}_2^+ \end{array} \right]  - \left[ \begin{array}{cc} b_{11} & b_{12} \\ b_{21} & b_{22} \end{array} \right] \left[ \begin{array}{c} \bar{\psi}_1^- \\ \bar{\psi}_2^- \end{array} \right] + \Sum_{m=1,4} \left[ \begin{array}{cc} c_{11} & c_{12} \\ c_{21} & c_{22} \end{array} \right] \left[ \begin{array}{c} L_{m1} \\ L_{m2} \end{array} \right] }

\item Setup global matrix equation,
\eqn{ \left[ \begin{array}{cccc} 
 [A] & [B] & [C] & [D] \\
 \left[E\right] & [1] & [F] & [G] \\
 \left[H\right] & [I] & [1] & [J] \\
 \left[K\right] & [L] & [M] & [1] \\
\end{array} \right]  
\left[ \begin{array}{c} 
[\bar{\psi} ] \\ 
\left[L_x\right] \\ 
\left[L_y\right] \\ 
\left[L_z\right] \end{array} \right] = \frac{1}{\keff} 
\left[ \begin{array}{cccc} 
[m] & [0] & [0] & [0] \\ 
\left[0\right] & [0] & [0] & [0] \\ 
\left[0\right] & [0] & [0] & [0] \\ 
\left[0\right] & [0] & [0] & [0] 
\end{array} \right] 
\left[ \begin{array}{c} 
[\bar{\psi} ] \\ 
\left[L_x\right] \\ 
\left[L_y\right] \\ 
\left[L_z\right] \end{array} \right] }

\item We solve the 3D balance equation by fission source and flux/leakage iteration, similar to finite difference, but leakate term has group-to-group coupling. 

\item Update $\keff$ and return to 2 until converged. 
\end{enumerate}

Next \textbf{the non-linear iterative solution} sequence was developed,
\begin{enumerate}
\item  Guess $\keff$, solve the standard 3D finite difference equations. 

\item Evaluate all interface currents from 2-node equations,
  \eqn{ \left[ \begin{array}{c} J_1 \\ J_2 \end{array} \right] = \left[ \begin{array}{cc} a_{11} & a_{12} \\ a_{21} & a_{22} \end{array} \right] \left[ \begin{array}{c} \bar{\psi}_1^+ \\ \bar{\psi}_2^+ \end{array} \right]  - \left[ \begin{array}{cc} b_{11} & b_{12} \\ b_{21} & b_{22} \end{array} \right] \left[ \begin{array}{c} \bar{\psi}_1^- \\ \bar{\psi}_2^- \end{array} \right] + \Sum_{m=1,4} \left[ \begin{array}{cc} c_{11} & c_{12} \\ c_{21} & c_{22} \end{array} \right] \left[ \begin{array}{c} L_{m1} \\ L_{m2} \end{array} \right] }
It's almost finite difference, but higher order in the sense that we say if flux has already converged, what would the current be. 

\item Redefine equivalent coupling terms to preserve nodal solution for current. Notice the coefficients $\alpha, \beta, \gamma, \delta$ are non-linear functions of the solution. 
  \eqn{ \left[ \begin{array}{c} J_1 \\ J_2 \end{array} \right] = \left[ \begin{array}{cc} \alpha & 0 \\ 0 & \gamma \end{array} \right] \left[ \begin{array}{c} \bar{\psi}_1^+ \\ \bar{\psi}_2^+ \end{array} \right]  
- \left[ \begin{array}{cc} \beta & 0 \\ 0 & \delta \end{array} \right] \left[ \begin{array}{c} \bar{\psi}_1^- \\ \bar{\psi}_2^- \end{array} \right] }

\item Set up global matrix equation,
  \eqn{ [A] [[\bar{\psi}]] = \frac{1}{\keff} [M] [[\bar{\psi}]] }

\item Solve 3D balance equations by fission source and flux iteration (same as finite-difference, and no group-to-group spatial coupling term). 

\item Update $\keff$ and return to 2 until converged. 
\end{enumerate}
This method is almost the same as CMFD, but with slightly different coefficients $\alpha, \beta, \gamma, \delta$. The point is, whether we do any one of the three nodal methods, we get a slightly different net current, the rest of the iteration looks the same. Mechanically, it is the same as the two-node two-group finite difference solution that we've already done. 



\clearpage
\topic{Summary of Transverse Integration Methods}
\begin{enumerate}
\item Transverse integration method is an innovative way to solve 3D diffusion equation by converting 3D PDE into 3 ODEs: 
  \begin{itemize}
    \item Solutions are only weakly dependent on the leakage shapes;
    \item Transverse leakage is approximated by a second order polynomial and iteratively updated.
  \end{itemize}

\item NEM is simple and efficient; it facilitates multi-group calculations, but loses accuracy for highly varying flux problems. 

\item ANM has the best accuracy, but it is not easy (though has been done) for general multi-group problems.

\item SANM is the most practical for LWR applications, because it has simple algebra, is multi-group applicable, and is comparable in accuracy to ANM. 

\item Non-linear solution method (eg, iterative) can be easily applied to all three methods. 
\end{enumerate}



\end{document}

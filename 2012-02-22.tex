\documentclass{school-22.211-notes}
\date{February 22, 2012}

\begin{document}
\maketitle


\clearpage
\topic{Resonance Integrals vs. Group Cross Section}
We introduce \hi{(Infinite) Resonance Integrals} as flux weighted (that is, weighted with 1/E spectrum $\phi(E) \sim \frac{1}{E}$) microscopic cross section: 
\begin{align}
\RI &=  - \int_{u1}^{u2} \sigma(u) \du \\
u &= \ln \left( \frac{E_0}{E} \right) = \ln(E_0) - \ln E \\
\du &= - \frac{1}{E} \dE \\ 
\RI &=  \int \sigma (E) \frac{1}{E} \dE 
\end{align}
Notice, 
\begin{itemize}
\item RI is defined directly from cross section data; no flux calculation is required because 1/E spectrum is assumed; 
\item RI depends on the normalization of the 1/E flux spectrum; it is also implicitly assumed that flux equals 1 when $E = 1$ through normalization;
\item RI is independent of energy bounds for isolated resonances;
\item RI is independent of temperature. This is because if the energy bound is larger enough, then as temperature increases, the spectrum would broaden, but because the area under the curve remains the same, assume the cross section is constant, then RI is essentially integrating the spectrum, which would not change upon temperature change;
\item RI is useful for inter-comparing libraries or cross section models. It is a classic way to evaluate new resonance data typically from 0.5 eV to 10 keV. We use RI to check our resonance data, in particularly the three big resonances at 6.67, 21, 26 eV. Numerical test of the SLBW RIs shows that the RI comes out to be within 1\% of ENDF/B-VII Reich-Moore data (Lec 6, slide 17, with 0.01 eV as spacing in histogram). 
\end{itemize}

\hi{Group cross section} is a similar but much more useful quantity. From its definition, we see $\sigma_g$ does not depend on the flux normalization: 
\begin{align}
\sigma_g &= \frac{\int_{E_1}^{E_2} \sigma(E) \phi(E) \dE }{\int_{E_1}^{E_2} \phi(E) \dE} 
\end{align}
If we make assumptions on the flux spectrum, then we can relate $\sigma_g$ to $\RI_{\eff}$,
\begin{align}
\phi(E) &\sim \frac{1}{E} \\
\sigma_g &= \frac{\int_{E_1}^{E_2} \sigma(E) \frac{1}{E} \dE }{\int_{E_1}^{E_2} \frac{1}{E} \dE} 
= \frac{\RI_{\eff}}{\ln(E_2) - \ln(E_1)}  
= \frac{\RI_{\eff}}{\ln(E_2/E_1)} \\
\Aboxed{\RI_{\eff} &= \sigma_g \ln(E_2/E_1) } \label{RIeff}
\end{align}
Notice\footnote{Review here for exam}:
\begin{itemize}
\item Group cross section by definition depends on both cross section and flux spectrum. 
\item Group cross section depends on the flux, but not on the normalization of flux (that is, only the shape matters, not the magnitude);
\item Group cross section depend explicitly on energy bounds (widths) of the groups; 
\item Effective RI can be computed from group cross sections and group energy bounds as in Eq.~\ref{RIeff}; As spectrum approaches 1/E, the effective RI computed from group cross sections will approach infinite RI. 
\end{itemize}


\clearpage
\topic{Temperature Effects On Cross Section (Doppler Broadening)}
\hi{Doppler Broadening} means, as the temperature increases, the width of the spectrum decreases, though the area under the curve stays the same. There are two possibilities: 
\begin{enumerate}
\item Infinite RI (infinite dilution factor) is independent of temperature becase the area under the psi chi curves are constant. 
\item Effective RI (finite concentration of \ce{^{238} U}) is dependent of temperature because of the self-shielding effect: as temperature increases, $\RI_{\eff}$ would increase. Also, as U/H increases, the fraction change of $\RI_{\eff}$ would increase as well as illustrated in Figure~\ref{Doppler}. That is, \textit{The higher the concentration of U238, the larger the Doppler temperature effects become.}
\end{enumerate}

In general, doppler broadening is the broadening of spectral lines due to the Doppler effect caused by a distribution of velocities of atoms or molecules. There are a couple of different kinds of Doppler broadening, including the thermal Doppler broadening due to the thermal motion of the particles; also, there are broadening depends on the frequency of the spectral line, the mass of the emitting particles, and the temperatures. 

Reference 1: Reuss Section 8.4. 

Reference 2: Handbook of Nuclear Engineering Chapter 4 Section 3. 

\begin{figure}
  \centering
  \subfloat[U/H=0.01]{\label{UH0.01}\includegraphics[width=0.5\textwidth]{images/Doppler-RI-1.png}}
  \subfloat[U/H=0.1]{\label{UH0.1}\includegraphics[width=0.5\textwidth]{images/Doppler-RI-2.png}}  
  \caption{Impact of Temperature On Effective RI/Resonance Absorption, Doppler Broadening} \label{Doppler}
\end{figure}

\clearpage
\topic{Energy Self-Shielding Effects On Spectrum}
As we increase the U/H ratio, the RI decreases in the three big resonance regions, and we see big dips on the spectrum plot. When U/H = 1.0, the spectrum is distorted. There are so much U238 in the fuel, that there is no flux in the fuel anymore. As we increase the number of uranium atoms by a factor of 10, the number of absorption per atom is decreased by a factor of 3. That is, the total aborption still increases, but the absorption per atom decreases. Figure~\ref{self-shielding} illustrates that RI are very dependent on the density of resonant materials. 
\begin{figure}
  \centering
  \includegraphics[width=4in]{images/self-shielding.png}
  \caption{Impact of Energy Self-Shielding On RI} \label{self-shielding}
\end{figure}

Self-shielding/flux depression: 
\begin{table}
  \centering
  \begin{tabular}{|l|p{6cm}|p{6cm}|} \hline
    & Observation & Explaination \\ \hline
    Spatial 
    & if natural uranium was made into a lump surrounded by moderator, resonance absorption $\down \down$  
    & U238 in the center of a pin is `shielded' from neutrons in the moderator \\ \hline
    Energy
    & if U/H $\up$, absorption rate per atom (for neutrons slowed down in a water \& U mix) $\down$   
    &  \\ \hline
  \end{tabular}
  \end{table}



\clearpage
\topic{Spectral Hardening Shifts Resonant Absorption Rates}
\textit{As more absorber (uranium in this case) is added, the higher energy ranges become more important (more absorption happens in the higher energies), and the peak of the thermal spectrum shifts to the right.} Figure~\ref{spectral-hardening} also illustrates that self-shielding reduces the large resonance absorption fractions. 1/v absorption of U238 become more significant. 
\begin{figure}
  \centering
  \includegraphics[width=4in]{images/spectral-hardening.png}
  \caption{Impact of Spectral Hardening On RI} \label{spectral-hardening}
\end{figure}


\end{document}

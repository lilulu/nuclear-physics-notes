\documentclass{school-22.101-notes}
\date{December 7, 2011}

\begin{document}
\maketitle

\topic{Charged Particle Interaction}
We want to calculate stopping power $- \dTdx$ which is a way to represent the energy deposition of the system. We are going to talk about two types of iteractions: 
\begin{enumerate}
  \item Collision/ionization. 
  \item Radiation (bremsstrahlung). 
\end{enumerate}

\subtopic{Collision/Ionization}
We start by drawing the diagram Fig.~\ref{cp-collision}. We know that the potential is a coulomb potential $V \sim \frac{ze^2}{r}$, then the force must be $\propto \frac{1}{r^2}$ as $F = - \gradient V$. We expect $F_y = F \cos \theta, \int \dt F_x(t) = 0$, that is, the momentum/force imposted to $e$ only in $F_y(t)$. Further consider time as $\dt = \frac{1}{v} \dx$. 
\begin{figure}[ht]
  \centering
  \includegraphics[width=4in]{images/ni/cp-collision.png}
  \caption{Collision Cylinder for Deriving the Energy Loss} \label{cp-collision}
\end{figure}

\begin{align}
    P_e &= \int \dt \overbrace{\frac{Ze^2}{x^2 + b^2}}^{F} \overbrace{\frac{b}{\sqrt{x^2 + b^2}}}^{\cos \theta} \\
    &= \int_{-\infty}^{\infty} \frac{2e^2}{x^2+b^2} \frac{b}{\sqrt{x^2 + b^2}} \frac{\dx}{v} \\
    &\approx \frac{2e^2 b}{v} \int \frac{\dx}{(x^2 + b^2)^{3/2}} = \frac{2Ze^2}{vb} \\
    \frac{P_e^2}{2m} &= \frac{2 (Ze^2)^2}{m_e b^2 v^2} 
\end{align}
Then we can write the energy loss per distance $\dTdx$, using that $2 \pi b \derivative b \dx$ is the element volume, $\left(\frac{Zze^2}{v b} \right)^2 \frac{1}{2m_e}$ is the energy loss, $n$ is the number density in \#/$\cm^3$, and $nZ$ is the number of neutrons per volume, 
\begin{align}
    \mbox{Stopping Power} &= - \frac{\dT}{\dx} = \int_{b_{min}}^{b_{max}} (2 \pi b \derivative b) \frac{1}{2m_e} \left( \frac{Zze^2}{vb} \right)^2 (n Z) \\
    &= \frac{4 \pi (Ze^2)^2nZ }{m_e v^2 } \ln \left( \frac{b_{max}}{b_{min}} \right) = \frac{4 \pi Z^2 e^4 n Z}{m_e v^2} \ln \left( \frac{2 m_e v^2}{\bar{I}} \right) 
\end{align}
A word about the $\frac{b_{max}}{b_{min}}$: we need to estimate this term; fortunately, a log function changes slowly, so we do not care all that much about the accuracy of this term. We estimate it as $\frac{2m_e v^2}{I}$. 

Relativistic Stopping Power, \hi{`Bethe Formula'} (correction for high energy):
\eqn{\left( -\dTdx \right)_{\mathrm{coll}} = \frac{4 \pi Z^2 e^4 nZ}{m_e v^2} \left[ \ln \left( \frac{2 m_e c^2 \beta^2}{I (1-\beta^2) } \right) - \beta^2 \right]   }
collision with nucleus: $\left( -\dTdx \right)_{\mathrm{coll}}$ increases by factor of Z; decreases by factor of $\frac{m_e}{M[Z]}$. 

The take-away is, 
\eqn{ \Aboxed{ \left( -\dTdx \right)_{\mathrm{coll}} &\propto \frac{nZ}{m} = \frac{Z}{m_e} } }


%%%%%%%%% Radiation %%%%%%%%
\subtopic{Radiation Loss}
The radiation loss is not all that interesting; but the range is important! We start by defining radiation loss as emission of x-rays due to sudden deceleration of charged particles. We do a hand-wavy argument about intensity (which is always the square of another quantity): 
\eqn{ I \sim (ze \times \mathrm{acceleration})^2 \sim \left( \frac{zeZe}{m} \right)^2 }
That is, \hi{radiation loss is important for high Z absorber with a small mass (e.g., electrons matter, radiation loss does not matter for heavy charged particles).} 
After derivation (see SY6), we arrive at 
\eqn{ \left( - \dTdx \right)_{\mathrm{rad}} = n \int_0^T \d(hv) hv \left[ \frac{\dsigma}{\derivative (hv)} \right]_{\mathrm{rad}} = n (T+ m_e c^2) \sigma_{\mathrm{rad}} \boxed{\sim (T + m_e c^2) }  } 





\subtopic{Range-energy Relation}
We can write out the definition for range, manipulate it, and apply Bethe curve (of the $\frac{\dT}{\dx}$ vs. $T$ relation). 
\begin{align}
    R &= \int_0^R \dx = \int_{E_0}^0 \frac{\dx}{\dE} \dE = \int_0^{E_0} \left( - \frac{\dE}{\dx} \right)^{-1} \dE  \approx \begin{dcases*}
 \int_0^{E_0} E \dE = E_0^2 & $T_0 < Mc^2$, in the $\frac{1}{v^2}$(or say $\frac{\dT}{\dx} \sim 1/T$) range \\ 
\int_0^{E_0} \dE  = E_0 & $T_0 > Mc^2$, in the log/real range. 
\end{dcases*}
\end{align}
Take-away:
\begin{enumerate}
\item A good rule of thumb to remember about comparing two particles' range:
\eqn{ \frac{R_1 (v)}{R_2 (v)} = \frac{Z_2^2 M_1}{Z_1^2 M_2} }

\item Range of validity:
\begin{itemize}
    \item $E \gg 500 \bar{I}$. 
    \item $\frac{Z e^2}{\hbar v} = \left( \frac{e^2}{\hbar c} \right) \frac{Z}{v/c} = \frac{Z}{137 (v/c) } \ll 1 $
\end{itemize}

\item Know Fig. 14.7.
\begin{figure}[ht]
  \centering
  \includegraphics[width=5in]{images/ni/14.7.png}
  \caption{Range-energy Relation for Electrons in Al (SY6 Fig. 14.7, Evans p.624)} \label{14.7}
\end{figure}


\item Experimentally one can determine the energy loss by the number of ion pairs produced from an ionization event. The number of ion pairs, $i$, produced per unit path length is called \hi{Specific ionization}: 
\eqn{ i(x) = \frac{1}{W} \left( -\frac{\dT}{\dx} \right)_{coll} }
where $W$ is the energy required for a particle of certain energy to produce an ion pair (and this is the term we know). 

\item Specific ionization leads to Bragg curve (see Fig.~\ref{14.2}, \ref{14.3}). Residual range refers to the distance still to travel before coming to rest. Proton range is 0.2 cm shorter than that of $\alpha$-particle. Fig.~\ref{14.2} says that the number of ion pairs per unit path for a single proton and a single alpha particle as a function of residual range. The residual range is the distance left to travel until the particle come to rest. The horizontal scale is such that on the left part of the diagram both particles have similar speeds. The proton range then is 0.2 air-cm shorter than the alpha-particle range. 
\begin{figure}[ht]
  \centering
  \includegraphics[width=5in]{images/ni/14.2.png}
  \caption{Specific Ionization of Heavy Particles in Air (SY6 Fig. 14.2, Meyerhof p.80)} \label{14.2}
\end{figure}

\begin{figure}[ht]
  \centering
  \includegraphics[width=5in]{images/ni/14.3.png}
  \caption{Specific Ionization for An Individual Particle vs. Bragg Curve (SY6 Fig. 14.3, Evans p.666)} \label{14.3}
\end{figure}


\item To determine the range of charged particle, we do a transmission experiment: the mean range $\bar{R}$ is defined as the absorber thickness at which the intensity is reduced to 1/2 of the initial value; the extrapolated range $R_0$ is obtained by linear extrapolation at the inflection point of the transmission curve. 

Recall we use to think $\displaystyle I(x) = I_0 e^{-\mu x}$ where $e^{-\mu x}$ is the probability of penetration, $\mu$ is the attenuation coefficient which is 1/distance and could be $\Sigma = N \sigma$. In this case $\frac{I}{I_0}$ is not exponential. Thus in charged particlesinteractions it is not sufficient to think of $I(x)$; one should be thinking about the range R. 
\begin{figure}[ht]
  \centering
  \includegraphics[width=5in]{images/ni/14.4.png}
  \caption{Determination of Range by Transmission Experiment (SY6 Fig. 14.4, Knoll)} \label{14.4}
\end{figure}
\end{enumerate}



%%%% Mass Absorption %%%%
\subtopic{Mass Absorption}
\eqn{ - \dTdx \sim nZ = \frac{\rho N_0}{A} Z = \rho N_0 \frac{Z}{A} }
If we consider $\frac{Z}{A}$ to be a constant, then this seems to imply that stopping power divided by density is a constant for all matter. That is, if we define $\dw = \rho \dx$, then $-\frac{\dT}{\dw}$ should be constant. See Fig.~\ref{14.1}. 
\begin{figure}[ht]
  \centering
  \includegraphics[width=5in]{images/ni/14.4.png}
  \caption{Mass Absorption Energy Losses for Electrons in Air, Aluminum, and Lead (SY6 Fig. 14.1)} \label{14.1}
\end{figure}




%%%%%%%%%%% 
\subtopic{Cerenkov Radiation}
Typically, $v_{light} = \frac{c}{n}.$ For instance, speed of light in water is 0.75c. When $v> \frac{c}{n}$, we see the blue glow in ractors that are water cooled. 



\subtopic{Summary}
\begin{enumerate}
\item Compare the electron curve to the proton curve, it looks like the proton curve is just electron curve shifted to the right (shifted to higher energies). 
\item Know that $m_p = 2000 m_e$, so for the same set-up, the collision loss for protons is 2000 times that of electrons. 
\item It is safe to ignore radiation loss for protons; whereas for electrons we have to consider both collision loss and radiation loss. 
\end{enumerate}

\begin{figure}[ht]
  \centering
  \includegraphics[width=5in]{images/ni/lead-ep.png}
  caption{Electron and Proton Energy Loss in Lead}
\end{figure}


%\begin{figure}
%   \centering
%   \includegraphics[width=4in]{images/ni/fusion-mechanism.png}
%   \caption{Mechanism for Fusion\label{fusion-mechanism}}
%\end{figure}



\end{document}

\documentclass{school-22.101-notes}
\date{November 7, 2011}

\begin{document}
\maketitle







%%%%%%%%%%%%%%%% Spin-orbit Coupling %%%%%%%%%%%%%
\topic{Spin Orbital Coupling} \label{spin-orbit-coupling}
\subtopic{Potential with Spin-Orbit Coupling}
Our previous attempts suggest that single particle interaction is missing something fundamental. Thus we add the interaction between the orbital angular momentum and the intrinsic spin angular momentum of nucleon: 
\eqn{ V_{\mathrm{nuc}} (r) = V_0 (r) + V_{\mathrm{so}} (r) \frac{\lhat \cdot \shat}{\hbar^2} }
Notice:
\begin{itemize}
\item $\lhat, \shat$ are operators on a single nucleon; 
\item the $\frac{\lhat \cdot \shat}{\hbar^2}$ term could give the proper separation of the sub-shells;
\item Adding spin-orbit coupling does not destroy the physical content of the potential as the intermediate potential is a good guess for how the nuclear potential should look like; 
\item \textcolor{blue}{the interaction is NOT spherically symmetric.}
\end{itemize} 

\begin{align}
\expect{\lhat \cdot \shat} &= \frac{\hbar^2}{2} \left[ j(j+1) - l(l+1) - \frac{3}{4} \right] =
\begin{dcases*}
- \frac{\hbar^2}{2} (l+1) & for $j = l-\frac{1}{2}$. \\ 
\frac{\hbar^2}{2} l & for $j = l + \frac{2}{2}$. 
\end{dcases*} 
\\
V_{\mathrm{nuc}} (r) &= 
\begin{dcases*}
V_0 - \frac{l+1}{2} V_{\mathrm{so}} & for $j = l-\frac{1}{2}$. \\ 
V_0 + \frac{l}{2} V_{\mathrm{so}} & for $j = l + \frac{2}{2}$. \\
\end{dcases*}
\end{align}
Keep in mind that $V_0 < 0, V_{\mathrm{so}} < 0$, so 
\begin{itemize}
\item $j = l+\frac{1}{2}$: pushes the well and the energy levels down (lower), more tightly bound; the attractive well is more attractive; 
\item $j = l - \frac{1}{2}$: pushes the well and the energy levels higher (up), more weakly bound.  
\end{itemize}

\subtopic{Consequences of Spin-orbit Coupling}
\begin{enumerate}
\item Each of the original $l$ level would be spitted into two states $j= l \pm \frac{1}{2}$ as in Figure~\ref{s-o-coupling}. Notice the total number of states preserved even though degeneracy is not based on $m_l, m_s$ in this new notation. We use spectroscopic notation $nl_j$.  
\begin{figure}[h!]
    \centering
    \includegraphics[width=4.6in]{images/shell/spin-orbit-coupling.png}
    \caption{Spin-Orbit Coupling, Splitting the Original $l$ Level into Two States}
    \label{s-o-coupling}
\end{figure}

\item The $j=l-\frac{1}{2}$ state makes the well shallower, making the state more weakly bound; the $j=l+\frac{1}{2}$ staet makes the well deeper, making the state more tightly bound. 
\begin{figure}[h!]
    \centering
    \includegraphics[width=1.5in]{images/shell/spin-orbit-coupling-potential.png}
    \caption{Potential Shifts after Applying Spin-Orbit Coupling}
\end{figure}

\item For a pair of states with $l>0$, the energy splitting difference increases with increasing $l$:
\eqn{ \Delta E = \expect{ \lhat \cdot \shat }_{j = l+\frac{1}{2} } - \expect{ \lhat \cdot \shat }_{j = l-\frac{1}{2} } = \frac{\hbar^2}{2} (2l+1) }
The physical interpretation of the this is that states with larger $l$ values split more; for instance, $1f$ state split more than than $2p$. In the deuteron example, $l=0$, hence we do not consider spin-orbit coupling. 
\end{enumerate}


\subtopic{Correction to the Intermediate Form Potential Shell Model}
\textcolor{blue}{Splitting of highest $l$ (recall larger $l$ results in larger $\Delta E$) at each oscillator level leads to re-joining of new j-levels with the top of the lower shell, and accounts for all the magic numbers.} Even the non-existing 184 is predicted. This founding leads to a Nobel Prize. 
\begin{enumerate}
\item $n=3$ oscillator level, the induced $1f_{7/2}$ reduces 8 nucleons from the upper shell, and makes a new `intermediate' shell with 8 nucleons ($D(1f_{7/2}) = 2\times j + 1 = 8$), which account for the magic number of 28 (=20+8). See Figure~\ref{shell-28}.
\begin{figure}
    \centering
    \includegraphics[width=3.5in]{images/shell/magic-number-28.png}
    \caption{Correction to Shell Model $n=3$\label{shell-28}}
\end{figure}

\item $n=4$ oscillator level: Figure~\ref{shell-50}.
\begin{figure}
    \centering
    \includegraphics[width=4in]{images/shell/magic-number-50.png}
    \caption{Correction to Shell Model $n=4$\label{shell-50}}
\end{figure}

\item $n=5$ oscillator level: Figure~\ref{shell-82}.
\begin{figure}
    \centering
    \includegraphics[width=4in]{images/shell/magic-number-82.png}
    \caption{Correction to Shell Model $n=5$\label{shell-82}}
\end{figure}

\item $n=7$ oscillator level: Figure~\ref{shell-126}.
\begin{figure}
    \centering
    \includegraphics[width=4in]{images/shell/magic-number-126.png}
    \caption{Correction to Shell Model $n=7$\label{shell-126}}
\end{figure}
\end{enumerate}

\uline{Summary}
\begin{enumerate}
\item Know what $l$ values each letter correspond to: $ f \to l=3, g \to l=4$ (starting from $l=0$, the notation goes like: s, p, d, f, h, g). 
\item After knowing $l$, the newly split states are $j= l \pm \frac{1}{2}$, write the new states out as original notation$_{j}$;
\item Degeneracy:$2 j+1$ degeneracy.  
\end{enumerate}



\end{document}
